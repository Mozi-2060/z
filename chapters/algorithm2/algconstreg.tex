\label{alg:constreg}

This algorithm concerns itself with producing a shadow of $T''$ from $\pi''$.
Because we do not refer to the inital triangulation or its projection in this algorithm or the next, we will abbreviate $T''$ by $T$, $\pi''$ by $\pi$, and $G''$ by $G$ to ease notation.
We begin by constructing an empty shadow $S$ and populating it with regions.
Each region $R$ of $G$ is a polygon, i.e.\ a topological disc, whose boundary is a triangulated $\sone$.
The edges of this $\sone$ are all associated with edges of $G$, and each edge of $G$ is coloured $I$ if the edge might produce an internal edge of our shadow, $F$ if the edge will produce a boundary edge of our shadow, or $H$ if the edge was introduced by removing an internal vertex of $G$.
Similarly, the regions of $G$ are coloured $I$ for ``internal'' and $H$ for ``hole.''

Let $p$ be a generic point in an internal region $R$.
The preimage of the point $p$ is the disjoint union $l_1\cup\dots\cup l_m$ where each $l_i$ is a piecewise--linear circle in $T$.
To split $\pi$ into $g\comp h$ with $g$ finite--to--one, we just say that $p$ pulls back through $g$ to exactly $m$ points.
The points in the regions of our shadow represent circles in $T$, the points in the interior edges of our shadow represent two circles joining together to become a single circle, the points in the boundary edges of our shadow represent circles collapsing to points, and the vertices of our shadow are interpreted using two local models.

We restrict our consideration to the internal regions of $G$.
The region $R$ pulls back through $\pi$ to a disjoint collection of $m$ open solid tori in $T$, and any torus $N$ in this collection can be seen as a circle bundle $N\overset{\pi}{\to} R$.
We pull $R$ back through $g$ to $m$ disjoint copies of itself.
We let a \emph{copy of $R$} be the pair $(P(R),l_i)$ where $P$ is a polygon identical to $R$ whose edges are coloured identically to the edges of $R$ and $l_i$ is one of the piecewise--linear circles of $T$ in the preimage of $p$.
For every internal region $R$ of $G$ and every piecewise--linear circle $l_i$ of $R$, we add $(P(R),l_i)$ to $S$.

We analyzed how the edges of $G$ are pulled back to $T$ through $\pi$ in Section~\ref{alg:countwedge} and this analysis can be extended to see how edges of $G$ are pulled back to our shadow through $g$.
Let $e$ be an internal edge of $G$ and let $A$, $B$ be the internal regions of $G$ which are adjacent over $e$.
There is an associated edge $E$ of $T$ so that $E$ projects over $e$ through $\pi$.
Let $a$ in $A$ and $b$ in $B$ be points near $e$ and let $\gamma_t = ta + (1-t)b$ be the line segment in the plane that connects $a$ and $b$, and let $t'a+(1-t'b)$ be the point on this line segment that lies in $e$.
We know that $\pi\inv(a)$ consists of exactly $m$ piecewise--linear circles $k_1\cup\dots\cup k_m$ and $\pi\inv(b)$ of exactly $n$ piecewise--linear circles $l_1\cup\dots\cup l_n$ in $T$.
The wedge number of $E$ is at most 2, so the numbers $m$, $n$ differ by at most 1.
Recall that piecewise--linear circles in $T$ are represented by ordered lists of 2--cells in $T$.
If a circle $k_i$ does not contain $E$ in any of its 2--cells, then the circle $k_i$ is unchanged as we pass over $e$.
There is a circle $l_j$ that is represented by the same ordered list of 2--cells (up to a cyclic permutation) as $k_i$.
We glue the region copies $(P(A),k_i)$ and $(P(B),l_j)$ over the edge of each corresponding to $e$.
The result is a copy of $A\cup_e B$, and the polygon $P(A\cup_e B)$ is identical, in terms of edge identifications, to the polygon obtained by gluing $A$ to $B$ over the edge $e$ in an orientation preserving way.
The piecewise--linear circles associated to the region copies $(P(A),k_i)$ and $(P(B),l_j)$ are equivalent, so we may take either circle as the piecewise--linear circle associated to $P(A\cup_e B)$.
All copies of $A$ and $B$ whose associated piecewise--linear circles not containing $E$ are dealt with in this manner.
We reduced all wedge numbers to at most 2 via edge blowups, so there are exactly three cases to consider when dealing with the copies of $A$ and $B$ containing $E$.

If the wedge number of $E$ is 0, then there is exactly one region copy $(P,l)$ among the copies of $A$, $B$ where $l$ contains $E$.
The edge copy in $P$ corresponding to $e$ is coloured $F$ for \emph{false}.
This edge corresponds to a shrinking singularity of $\pi$, where a circle in $T$ collapses to a point.

If the wedge number of $E$ is 1, then there are exactly two region copies $(P(A),k)$ and $(P(B),l)$ among the copies of $A$ and $B$ where $k$ and $l$ each contain $E$.
We combine $P(A)$ and $P(B)$ as before into $P(A\cup_e B)$, but the piecewise--linear circles $k$ and $l$ are distinct in a way that has been hinted at before.
The intersection of $k$ and $l$ as a list of 2--cells consists of exactly the 2--cells in either circle that do not contain $E$, and the symmetric difference of $k$ and $l$ is exactly the set of 2--cells in $T$ containing $E$.
We are left with the region copy pair $(P(A\cup_e B), (k,l))$.
If the region copy $P(A\cup_e B)$ is considered in the future, it is with respect to some edge $f$ of $A\cup_e B$ and $f$ is exclusively an edge of either $A$ or $B$.
The representative circle for $P(A\cup_e B)$ is then taken to be $k$ if $f$ is from $A$, and $l$ if $f$ is from $B$.
In this case, the edge of $G$ did not correspond to a singularity of $\pi$ at all.

If the wedge number of $E$ is 2, then there are exactly three region copies.
Without loss of generality, two of these are copies of $A$ and the last is a copy of $B$ and we call these copies $(P(A),k_1)$, $(P(A),k_2)$ and $(P(B),l)$.
In this case we have encountered the boundary of all three region copies along an internal edge of $S$, so we glue each region copy along their edge $e$.
This gluing should be orientation reversing when restricted to $(P(A),k_1)$, $(P(A),k_2)$ and orientation preserving when restricted to $(P(A),k_1)$, $(P(B),l)$ and to $(P(A),k_2)$, $(P(B),l)$.
Colour the shared edge $e$ by $i$ for \emph{internal}.
This edge corresponds to a simple saddle singularity of $\pi$, where a single circle splits into two distinct circles.

Iterating over all adjacent internal region pairs produces a connected simple polyhedron $S$ with boundary a graph whose connected components are coloured either $F$ or $H$.
The false boundary is not considered until the next chapter, but we fill in each boundary component coloured $H$ with a simple block.






