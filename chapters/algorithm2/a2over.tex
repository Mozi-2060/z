The previous chapter produced a map $\pi'':T''\to \RR$, where $T''$ is a 3--manifold with only spherical boundary components.
In Section~\ref{sec:bounding} we discussed the construction of a Stein complex from a Morse 2--function $f$ from a 3--manifold to $\RR$.
The only two properties we needed from $f$ were that $f$ had compact fibers and that $f$ was generic.
Genericity was only used to determine how the fibers of $f$ joined over singularities, and the piecewise--linear analogue to the critical values of the smooth Morse 2--function $f$ are the vertices and edges of $(G'',p'')$.
Our map $\pi''$ has compact fibers, and the wedge numbers of Section~\ref{alg:countwedge}, each reduced to at most two in Section~\ref{alg:blowup}, provide the information needed to determine how the fibers of $\pi''$ behave near the edges and vertices of $(G'',p'')$.
This means we may construct a Stein complex for $T''$ from the data produced in Chapter~\ref{cha:projection}.
The data used to construct this complex is a planar graph $(G''',p''')$, lists of circles projecting over the regions of $(G''',p''')$, and a colouring and wedge number for each edge of $G'''$.
The Stein complex constructed this way is a shadow polyhedron of $T''$, as defined in Section~\ref{sec:shadow}.
By Lemma 4.4 of \cite{CostThur08}, the polyhedron constructed is standard, i.e.\ each region is a disc.
To obtain a shadow of $T''$, we equip each region not touching the boundary of the polyhedron with a gleam

