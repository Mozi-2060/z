Every 3--manifold is the boundary of some 4--manifold.
The earliest proofs of this fact are due to Ren\'e Thom in the early 1950's \cite{Thom}.
Since then the problem has been approached from multiple avenues.
A series of papers in the 1960's by Hirsch~\cite{Hirsch61}, Rokhlin~\cite{Rokhlin65}, and Wall~\cite{Wall65} built up the theory of immersing and then embedding 3--manifolds in $\R^5$.
The embedded 3--manifold then bounds a ``Seifert 4--manifold'' as with knots bounding Seifert surfaces in $\R^3$.
Also in the 60's, Lickorish showed that all orientable 3--manifolds may be represented as surgery on a 3--sphere, and thus bound 4--manifolds.
A later result by Rourke~\cite{Rourke85} demonstrates something similar, this time by investigating Heegaard diagrams of the given 3--manifolds.

Historical arguments prove 3--manifolds bound 4--manifolds, but we normally describe our 3-- and 4--manifolds using triangulations.
We would like a process that takes us from a triangulated 3--manifold to a triangulated 4--manifold.
The work done by Turaev in ~\cite{Turaev91} paved the way for an argument in ~\cite{CostThur08} that includes a process used to estimate a bound on the number of 4--dimensional simplices needed to triangulate a 4--manifold whose boundary is an input 3--manifold.
This process involved the definition of structures called \emph{shadows} that feature prominently in the work of Thurston and Costantino as the time.
A shadow is, essentially, a 2--dimensional CW--complex with some extra structure
Most importantly, a shadow puts restrictions on how many 2--cells may be incident to a single vertex or attached over a single edge.
The extra structure of the shadow is not needed in the scope of this work, so we define only the concept of the 2--dimensional CW--complex.

What follows is a brief overview of what to expect, only slightly more descriptive than our table of contents.
\begin{itemize}
	\item Chapter \ref{cha:manifolds} consists entirely of geometric topology fundamentals.
	Most of the material is used only to support the main topics of chapter 2, which are handle attachment and definition of triangulation.
	A reader that is well read in those topics is encouraged to skim Chapter \ref{cha:manifolds} in order to familiarize themselves with the notation used.
	\item Chapter \ref{cha:cobordisms} is a build up to an argument that 3--manifolds bound 4--manifolds in the smooth case.
	We make this argument in a constructive way so that it can be turned into an algorithm in the piecewise--linear case.
	The main focus is Section \ref{sec:3bound4}, and the first two sections cover definitions and the techniques of proof in the case of orientable 2--manifolds bounding 3--manifolds that we draw from in Section \ref{sec:3bound4}.
	\item Chapter \ref{cha:algorithm} is our collection of algorithms.
	We initialize some key definitions and present some minor results before diving directly into the algorithms.
\end{itemize}