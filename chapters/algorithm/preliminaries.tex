\label{def:preliminaries}
Before beginning the algorithm, we should establish some properties of piecewise--linear maps from polyhedra to $\RR$.
These properties are called upon to support some algorithms in this chapter.
We also review the basics of planar graphs, as these are the objects that encode the information we wish to carry to Chapter~\ref{cha:alg2}.

\begin{defn}
	\label{def:planargraph}
	Let $G$ be a connected graph such that an embedding of $G$ in $\RR$ exists.
	Fix $p:G\to\RR$ to be such an embedding.
	Then the pair $(G,p)$ is called a \emph{planar graph}.
	The image $p(G)$ separates $\RR$ into path connected components called the \emph{regions} of $(G,p)$.
	There is exactly one region of $(G,p)$ which is unbounded in the plane, and whose boundary is a cycle of $G$.
	This region, its boundary cycle and all edges and vertices in that boundary cycle are said to be \emph{outer}.
	All other other regions, vertices and edges are \emph{inner}.
	Regardless of the choice of embedding $p$, every chordless cycle of $G$ is the boundary of exactly one inner region of $(G,p)$, and every inner region of $(G,p)$ is bounded by exactly one chordless cycle of $G$, so the inner regions and chordless cycles of $(G,p)$ are in $1$--$1$ correspondence.
\end{defn}

\begin{defn}
	\label{def:stdproj}
	Let $T$ be a tetrahedron with the four vertices $u$, $v$, $w$, $x$, six edges $uv$, $uw$, $ux$, $vw$, $vx$, $wx$, and faces $\hat{u}$, $\hat{v}$, $\hat{w}$, $\hat{x}$, where a face is named by the vertex of $T$ it does not contain.
	Define a projection $\pi: T \to\DD$ by first choosing a map from the vertices of $T$ to distinct points in $\sone$.
	Each point $p$ of $T$ is decribed by the convex combination
	\[
		p = t_u u + t_v v + t_w w + t_x x
	\]
	with the $t_*$ nonnegative and summing to 1.
	We can define $\pi$ at $p$ by
	\begin{eqnarray}
		\label{affine_extension}
		\pi(p)
		&=&
		\pi(t_u u + t_v v + t_w w + t_x x) \nonumber \\
		&=&
		t_u \pi(u) + t_v \pi(v) + t_w \pi(w) + t_x \pi(x)
	\end{eqnarray}
	  
	Without loss of generality, we assume that the points $\pi(u),\pi(v),\pi(w),\pi(x)$ are ordered in a clockwise orientation about $\sone$.
	We call $\pi:T\to\DD$ a \emph{linear tetrahedral projection}.
\end{defn}

\begin{defn}
	\label{def:projpttypes}
	A point of $\DD$ in the image of $\pi$ is one of five types:
	\begin{enumerate}
		\item The four points of type 1 are the images of the vertices of $T$ under $\pi$.
		\item The single point of type 2 is the intersection $\pi(uw)\cap\pi(vx)$.
		\item All points in $\pi(uv)\cup\pi(vw)\cup\pi(wx)\cup\pi(xu)$, excluding the points of type 1, are of type 3.
		\item All points in $\pi(uw)\cup\pi(vx)$ excluding the points of type 1 or 2 are of type 4.
		\item The points of type 5 are the points outside of the image of any vertex or edge of $T$.
	\end{enumerate}
	The image of the $1$--skeleton of $T$ forms a planar graph $G$ inside of $\DD$ whose vertex set consists of the points of types 1 and 2 and whose edges consist of points of type 3 and 4.
	The planar graph embedding cuts the plane into five connected regions: four inner regions that contain all points of type 5, and and one outer region.
	This graph has four outer edges, consisting of the points of type 3, and four inner edges, consisting of the points of type 4.
	
	By definition, the preimage of a point of type 1 is a vertex of $T$.
	The preimage of the single point of type 2 is a line segment between the edges $uw$ and $vx$ interior to $T$.
	A point of type 3 is the image of exactly one point in an edge of $T$.
	A point of type 4 is in the image of exactly one edge and one face, so the preimage of one of these points is a line segment interior to $T$ between those two facets.
	Finally, points of type 5 are in the image of exactly two faces of $T$, and pull back to line segments interior to $T$ between those two faces.
\end{defn}
