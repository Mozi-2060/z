We enter this section with a piecewise linear analogue to the generic proper smooth map of Section \ref{sec:3bound4}, and focus on leaving it with a Stein complex to the map.
A generic proper smooth map already has a Stein complex that is described as a 2--complex, so this section is fairly straightforward and relatively light.
Only in Section \ref{sub:gleams}

The data structures we have to work with are:
\begin{enumerate}
	\item A solid polyhedral gluing $T'$ that is homeomorphic to $T$ minus some 3--balls.
	\item A piecewise--linear map $\pi':T'\to\C$ analogous to a generic proper smooth map.
	\item A 2--complex $T'*$ dual to $T'$.
	\item A 2--complex $X'$ generated by $\pi'$.
\end{enumerate}
We build the Stein complex for $\pi'$ in the order of increasing cells.

a 2--complex $X$

The properties this map must satisfy are laid out in Section \ref{sub:genericplmap}, and the algorithms used to ensure these conditions are laid out in the intermediate sections.

From $T$ a closed, orientable, edge--distinct 3--manifold triangulation, we obtain:

These are the data carried forward to Section \ref{sec:stein} that store the information used to construct a Stein complex for $\pi'$.




The previous chapter produced a map $\pi'':T''\to \RR$, where $T''$ is a 3--manifold with only spherical boundary components.
In Section~\ref{sec:3bound4} we discussed the construction of a Stein complex from a Morse 2--function $f$ from a 3--manifold to $\RR$.
The only two properties we needed from $f$ were that $f$ had compact fibers and that $f$ was generic.
Genericity was only used to determine how the fibers of $f$ joined over singularities, and the piecewise--linear analogue to the critical values of the smooth Morse 2--function $f$ are the vertices and edges of $(G'',p'')$.
Our map $\pi''$ has compact fibers, and the wedge numbers of Section~\ref{alg:countwedge}, each reduced to at most two in Section~\ref{alg:blowup}, provide the information needed to determine how the fibers of $\pi''$ behave near the edges and vertices of $(G'',p'')$.
This means we may construct a Stein complex for $T''$ from the data produced in Chapter~\ref{cha:projection}.
The data used to construct this complex is a planar graph $(G''',p''')$, lists of circles projecting over the regions of $(G''',p''')$, and a colouring and wedge number for each edge of $G'''$.
The Stein complex constructed this way is a shadow polyhedron of $T''$, as defined in Section\{deprecated\}.
By Lemma 4.4 of \cite{CostThur08}, the polyhedron constructed is standard, i.e.\ each region is a disc.
To obtain a shadow of $T''$, we equip each region not touching the boundary of the polyhedron with a gleam
