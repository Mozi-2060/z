We enter this section with a piecewise linear analogue to the generic proper smooth map of Section \ref{sec:3bound4}, and focus on leaving it with a Stein complex to the map.
A generic proper smooth map already has a Stein complex that is described as a 2--complex, so this section is fairly straightforward and relatively light.
Only in Section \ref{sub:gleams} do we delve into complicated subject matter.
As we are running fully parallel with Section \ref{sec:3bound4}, the arguments can be easily compared.

Section \ref{sec:proj} ended by providing us with some explicit data structures.
These structures are taken as given in the following sections.
What we have to work with:
\begin{enumerate}
	\item A solid polyhedral gluing $T'$ that is homeomorphic to $T$ minus some 3--balls.
	\item A piecewise--linear map $\pi':T'\to\C$ analogous to a generic proper smooth map.
	\item A 2--complex $T'^*$ dual to $T'$.
	\item A 2--complex $X'$ generated by $\pi'$.
\end{enumerate}
We build the Stein complex for $\pi'$ in the order of increasing cells.
Our 2--complex $X'$ corresponds completely to the image of $T'$ through $\pi'$, so the factorization of $\pi'$ should yield a 2--complex very similar to $X'$.
The 2--complex $S$ built in this section is the only data structure passed to the algorithms of Section \ref{sec:4man}.