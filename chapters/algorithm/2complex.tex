We establish some properties of piecewise--linear maps from polyhedra to $\C$, and develop an algorithm to produce an associated 2--dimensional CW--complex.
This CW--complex is used as a data type to keep track of the piecewise linear version of the generic proper smooth maps of Section \ref{sec:3bound4}.
In Section \ref{sec:stein}, we use the 2--complex produced to form a Stein complex.

\begin{defn}
	\label{def:stdproj}
	Let $\sigma$ be a tetrahedron with the four vertices $u$, $v$, $w$, $x$, six edges $uv$, $uw$, $ux$, $vw$, $vx$, $wx$, and faces $\hat{u}$, $\hat{v}$, $\hat{w}$, $\hat{x}$, where a face is named by the vertex of $\sigma$ it does not contain.
	Define a projection $\pi: \sigma\to\C$ by first choosing a map from the vertices of $\sigma$ to distinct points in $\sone$.
	Each point $p$ of $\sigma$ is described by the convex combination
	\[
		p = t_u u + t_v v + t_w w + t_x x
	\]
	with the $t_*$ nonnegative and summing to 1.
	We can define $\pi$ at $p$ by
	\begin{eqnarray}
		\label{affine_extension}
		\pi(p)
		&=&
		\pi(t_u u + t_v v + t_w w + t_x x) \nonumber \\
		&=&
		t_u \pi(u) + t_v \pi(v) + t_w \pi(w) + t_x \pi(x)
	\end{eqnarray}
	  
	Without loss of generality, we assume that the points $\pi(u),\pi(v),\pi(w),\pi(x)$ are ordered in a clockwise orientation about $\sone$.
	We call $\pi:\sigma\to\C$ a \emph{linear tetrahedral projection}.
\end{defn}

\begin{defn}
	\label{def:projpttypes}
	A point of $\C$ in the image of $\pi$ is one of five types:
	\begin{enumerate}
		\item The four points of type 1 are the images of the vertices of $\sigma$ under $\pi$.
		\item The single point of type 2 is the intersection $\pi(uw)\cap\pi(vx)$.
		\item All points in $\pi(uv)\cup\pi(vw)\cup\pi(wx)\cup\pi(xu)$, excluding the points of type 1, are of type 3.
		\item All points in $\pi(uw)\cup\pi(vx)$ excluding the points of type 1 or 2 are of type 4.
		\item The points of type 5 are the points outside of the image of any vertex or edge of $\sigma$.
	\end{enumerate}
\end{defn}

By definition, the preimage of a point of type 1 is a vertex of $\sigma$.
The preimage of the single point of type 2 is a line segment between the edges $uw$ and $vx$ interior to $\sigma$.
A point of type 3 is the image of exactly one point in an edge of $\sigma$.
A point of type 4 is in the image of exactly one edge and one face, so the preimage of one of these points is a line segment interior to $\sigma$ between those two facets.
Finally, points of type 5 are in the image of exactly two faces of $\sigma$, and pull back to line segments interior to $\sigma$ between those two faces.
Our classification of points in the image of $\pi$ naturally lends itself to the structure of a 2--complex.
We detail this construction in Algorithm \ref{alg:lintet2complex}.

\begin{algorithm}
	\caption{Generating a 2--complex from a linear tetrahedral projection}
	\label{alg:lintet2complex}
	\KwData{a simplex $\sigma$, a map $\pi:\sigma\to\C$}
	\KwResult{a 2--complex $X$}
	\Begin{
		$X^1 \longleftarrow \sigma^1$\;
		\While{there is a pair of edges $(e,f)=((u,v),(w,x))$ in $X$ that cross at the point $y\in\RR$}{
			delete $e$, $f$\;
			add a vertex $y'$ to $X$\;
			define $\pi(y')=y$\;			
		}
		\ForEach{chordless cycle $c=(v_1,v_2,\dots,v_k,v_1)$ in $X$}{
			attach a 2--cell over $c$\;
		}
	}
\end{algorithm}

For $T$ a 3--dimensional edge--distinct connected simplicial complex, the above algorithm breaks down when a triple of edges cross at the same point.
A sufficient condition to guarantee that a crossing point between a pair of edges is the crossing point of at most two edges (by the results in \cite{PoonRub98}) is if $\pi(T^1)$ is a subgraph of an odd dimensional complete graph $K_n$ in its standard position (i.e.\ with the $n$ vertices each at a distinct $n\nth$ complex root.)

\begin{defn}
	Let $T$ be a 3--dimensional edge--distinct connected simplicial complex and $\pi:T\to\C$ a map that is linear on each simplex of $T$.
	If, for some odd positive integer $n$, $\pi(T^0)$ is injective into the $n\nth$ roots of unity, then we say $\pi$ is a \emph{crossing distinct} projection of $T^1$ into $\C$.
\end{defn}

The 2--complex construction can now be extended to simplicial complexes.
In Algorithm \ref{alg:crossdis2complex} we continue to let $T$ be a 3--dimensional edge--distinct connected simplicial complex with crossing distinct projection $\pi:T\to\C$.

\begin{algorithm}
	\caption{Generating a 2--complex from a crossing distinct projection}
	\label{alg:crossdis2complex}
	\KwData{ $T$, $\pi:T\to\C$}
	\KwResult{a 2--complex $X$}
	\Begin{
		$X^1 \longleftarrow T^1$\;
		\While{there is a pair of edges $(e,f)=((u,v),(w,x))$ in $X$ that cross at the point $y\in\RR$}{
			delete $e$, $f$\;
			add a vertex $y'$ to $X$\;
			define $\pi(y')=y$\;			
		}
		\ForEach{chordless cycle $c=(v_1,v_2,\dots,v_k,v_1)$ in $X$}{
			attach a 2--cell over $c$\;
		}
	}
\end{algorithm}