




In the previous step we kept track of longitudes in the boundaries of triangulated open solid torii sitting in the boundary of our thickened handlebody $H$.
For a particular open solid torus $T$, take this curve to be $C_1$.
The building blocks of our $T$ allow us to easily craft another curve $C_2$ in the boundary of $T$ that has linking number $+n$ with respect to the 0--framing of $T$.
We construct a triangulated $\sthr$ from $T$ which has $T$ as a subtriangulation.
We do this by constructing another triangulated open solid torus $T'$ whose boundary has the same triangulation as the boundary of $T$.
We may choose any longitude of $T$ to be a meridian of $T'$, so we choose $C_2$.
When we glue $T$ to $T'$ along their shared boundary we obtain a triangulated $\sthr$.
This $\sthr$ has $T$ as a subtriangulation, and the curve $C_2$ in the boundary of $T$ bounds a disc in this triangulated $\sthr$.
We take $C_2$ to be a 0--framing of the core of $T$.
Finally, we may form $D^4$ as the cone of $\sthr$.
This $D^4$ has $T$ as a subtriangulation in its boundary $\sthr$ with 0--framed curve $C_2$.
Gluing $D^4$ to $H$ over $T$ takes the 0--framing curve of $T\subset \pd D^4$ to a $+n$--framed curve of $T\subset\pd H$.
We therefore have attached 2--handle to $H$ over the open solid torus $T$ with framing coefficient $+n$.
