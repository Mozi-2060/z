
In their paper ``3--Manifolds Efficiently Bound 4--Manifolds,'' Francesco C and Dylan Thurston demonstrated that $M$, a closed, orientable 3--manifold triangulated with $n$ simplices, bounds a 4--manifold $W$ triangulated using a number of simplices bounded by a quadratic polynomial in $n$.
First, they argue that 3--manifolds bound 4--manifolds.
Their argument builds a 4--manifold $W$ from a closed, oriented 3--manifold $M$ by filling in one boundary component of $M\times I$ with 4--dimensional 2--, 3--, and 4-- handles.
Attachment instructions for these handles are obtained by a Morse 2--function from $M$ to $\RR$.
A Morse 2--function is defined similarly to an ordinary Morse function.



A gleamed shadow of a 3-manifold M is a shadow of a 4-manifold bounded by M.
The ultimate goal of this thesis is to provide an algorithm that takes as input an edge-distinct triangulation $T$ of the closed orientable 3--manifold $M$ and produces as output a triangulation of a 4--manifold $W$ such that $\pd W$ is equivalent in the sense of triangulations to $T$.
This algorithm is split into three chapters.
Chapter \ref{cha:projection} builds a piecewise--linear Morse 2--function $T\to\RR$ and collects all data from that function necessary to build a gleamed shadow shared by $T$ and a 4--manifold bounded by $T$.
Chapter \ref{cha:shadow} builds a gleamed shadow from the information obtained in \ref{cha:projection}.
Chapter \ref{cha:manifold} assumes we have a gleamed shadow $(S,\glm)$ and constructs a triangulated 4--manifold whose shadow is $(S,\glm)$.

The data we are trying to collect and retain with the algorithms of this chapter are as follows:
\begin{enumerate}
  \item A list of polygons which we will call regions.
  \item A colouring of every edge of every region using one of the three colours $i,f,h$.
  \item A wedge number of $0$, $1$, or $2$ for any edge coloured $i$.
  \item A list of adjacencies between regions.
        An item in this list consists of a pair of regions and an $i$--coloured boundary edge of each region.
  \item A list of piecewise--linear circles associated to each region.
        An item in this list is an ordered list of polygonal 2--cells which live in a closed polyhedral 3--complex so that any two consecutive 2--cells lie on the boundary of the same 3--cell, yet any 3--cell contributes either zero or two 2--cells to any given list.  This list is indexed by the list of regions in item 1.
\end{enumerate}
This data is sufficient to build a shadow, as seen in Chapter \ref{cha:shadow}.
