\label{alg:blowup}

This algorithm takes as input an edge $e$ of the solid polyhedral gluing $T'$ with wedge number $w(e)$ at least three, and returns an altered solid polyhedral gluing $T''$ and an altered graph $G''$, where $e$ is replaced by a set of edges $e_i$, $i=1,\dots ,w(e)-1$, where each edge has wedge number exactly 2.
We blow up edges to reduce wedge numbers and we want to reduce all wedge numbers to at most 2.

To blow up the edge $e=xy$, we need to choose a pair of 3--cells over which $e$ will be blown up.
We need to slightly modify our definition of the link of an edge because we've truncated all of our tetrahedra into 3--cells, and some edges are going to be blown up.
Each tetrahedra has six edges and we can easily keep track of the associated edges in a truncated tetrahedron as well as a set of blown up edges in a blown up truncated tetrahedron.
An edge or set of edge copies in a tetrahedron or blown up tetrahedron can then be said to have an associated ``old'' edge, as it would appear in $T$.
We take $\lk{e}$ during this algorithm to be a triangulated circle $L$ whose edges are exactly the edges of the ``old'' link in $T$.
The main information contained in an ``old'' link is a string of crossing coefficients with respect to the edge $e$, and this information is not altered by the following algorithm.
The structure of a tetrahedron and the requirement that our initial triangulation $T$ was edge distinct means that a pair of ``old'' edges from $T$ can be opposite edges of at most one tetrahedron of $T$, hence a pair of such ``old'' edges uniquely determine a 3--cell of $T'$.

Because the wedge number of $e$ is at least 3, there are at least 6 edges of $\lk{e}$ with $+1$ crossing coefficients.
We choose a connected path $Q=q_0 \dots q_n$ in $\lk{e}$ whose crossing coefficients sum to exactly 4, and whose tail edges, $q_0 q_1$ and $q_{n-1}q_n$, each have $+1$ crossing coefficient.
The tail edges of this path along with the edge $e$ determine two 3--cells $t_0$,$t_1$ over which we blow $e$ up.
Two of the vertices of the $t_0$ are $x$ and $y$.

The image of $e$ under $\pi$ is the path $P=v^1\dots v^k$.
We choose to map $x$ to $v^1$ and $y$ to $v^k$.
The vertices of $G$ on the circle which are not $v^1$ or $v^k$ are partitioned into two sets which we arbitrarily designate as ``upper'' and ``lower'' with respect to $P$.
Subscripts containing $l$'s and $u$'s refer to lower and upper designations of the objects in question.
Without loss of generality, we'll say that the edge opposite $e$ in $t_0$ is $q_0 q_1$ and the edge opposite $e$ in $t_1$ is $q_{n-1} q_n$. 
Either way, the edge opposite $e$ in $t_{0,1}$ has crossing coefficient $+1$ with respect to $e$.
Because $Q$ has crossing sum exactly 4, the vertices $q_0$, $q_n$ are both either upper or lower.
We'll say that they're both lower, which forces $q_1$, $q_{n-1}$ to each be upper.
Blowing up the edge of a pair of solid polyhedra adds a face to each of the two solid polyhedron over which the two become glued.
The edge is called $e$ and has vertices $x$, $y$.
One of the edges that $e$ splits into forms an edge of two 2--cells, one containing $q_0$ and one containing $q_n$.
We call this edge $e_l$, and say that $e_l$ has vertices $x_l$, $y_l$.
The other edge that $e$ splits into forms the edge of two 2--cells, one containing $q_1$ and one containing $q_{n-1}$.
This edge is called $e_u$ and has vertices $x_u$, $y_u$.

Every vertex $v^i$ of $P$ is adjacent to exactly two other vertices $v_{ll}^i$ and $v_{uu}^i$ of $G$.
When $v^i$ is internal, each of $v^i$, $v_{ll}^i$, and $v_{uu}^i$ is on the path $P^i$, which is the image of an edge $e_i$ of $T$ through $\pi$.
Then $v_{ll}^i$ is the vertex on the path $P^i$ from $v^i$ to a lower vertex and $v_{uu}^i$ is the vertex on the path $P^i$ from $v^i$ to an upper vertex.
When $v^i$ is external, $v_{ll}^i$ and $v_{uu}^i$ are already upper and lower vertices, so $v_{uu}^i$ is the upper vertex and $v_{ll}^i$ is the lower vertex.

\{FIGURE: SAMPLE $P^i$ AND VERTICES AS DEFINED\}


\{FIGURE: BLOWUP ILLUSTRATED\}.

The blowup of $P$ is the graph $P\times I$ which gives us a copy of $P$ at $\{0\}$ and at $\{1\}$, called $P_l$ and $P_u$ respectively.
The new vertices are similarly subscripted, and the graph $P\times I$ also contains the edges $v_l^i v_u^i$ for every $i=1,\dots,k$.
Blowing up $P$ inside of $G$ means that we remove $P$ and all edges incident to $P$ from $G$.
We then add to $G$ the graph $P\times I$, and add the edges $v_{ll}^i v_l^i$ and $v_{ll}^i v_u^i$ for each $i$.
The embedding of $G$ incorporates $P\times I$ in the way depicted in the figure.

\{FIGURE: EMBEDDING OF $G$\}

The regions of $G$ incident to $P$ are pushed away from each other and new regions are created.
These new regions are bounded by the edges $v_u^i v_l^i$, $v_l^i v_l^{i+1}$, $v_l^{i+1} v_u^{i+1}$, $v_u^{i+1} v_u^i$ for $i=1,\dots,k-1$ and are the image of the new face $f$ of $T$.
The paths $P_l$ and $P_u$ and the edges $v_u^1 v_l^1$ and $v_u^k v_l^k$ are the image of the boundary edges of $f$.
Recall that these edges are $e_u$, $e_l$, $x_u x_l$ and $y_l y_u$.
We get that $x_l x_u$ and $y_l y_u$ map to $v_l^1 v_u^1$ and $v_l^k v_u^k$ so that lower and upper subscripts are mapped to each other.
This also makes $e_u$ map to $P_u$ and $e_l$ map to $P_l$.

The wedge numbers of $e_u$ and $e_l$ are exactly 2 and $w(e)-1$ respectively.
We can see this by saying that an edge of $\lk{e}$ contributes to the wedge number of $e_l$, (resp. $e_u$) if and only if that edge is adjacent to a 2--cell of $T'$ containing $e_l$, (resp. $e_u$).
If $w(e_l)$ is still greater than 2, we proceed by choosing another pair of 3--cells to blow $e_l$ up over.
This pair is found by extending $Q$ in both directions to $q_{-k}q_{-k+1}\dots q_0\dots q_n\dots q_{n+k'} q_{n+k'+1}$ where $q_{-k}q_{-k+1}$ and $q_n+k' q_{n+k'+1}$ each have crossing coefficient $+1$ with respect to $e_l$, and there are no other edges between $q_{-k}q_{-k+1}$ and $q_0 q_1$ and between $q_{n-1} q_{n}$ and $q_n+k' q_{n+k'+1}$ with crossing coefficient $+1$ with respect to $e_l$.
The residual triangulation structure leaves us with exactly two 3--cells, one containing $e_l$ and $q_n+k' q_{n+k'+1}$ and one containing $e_l$ and $q_{-k}q_{-k+1}$, over which to blow up $e_l$.
The algorithm for this blow up works for blowing up $e_l$ into $e_lu$ and $e_ll$, where $e_lu$ has wedge number 2 and $e_ll$ has wedge number $w(e_l)-1$.
Iterate until $e$ has been blown up into exactly $w(e)-1$ edges whose wedge numbers are exactly 2.

There are a few new regions in $G$, and we need to know how generic points in those regions are pulled back into $T''$ by $\pi''$.
Fortunately, we are able to derive the mapping circles of a region of $G''$ entirely from the mapping circles of an adjacent region.
Let $R$ be a new region of $G''$ whose mapping circles are unknown, and let $R'$ be a region of $G''$, adjacent to $R$ over the edge $e$ of $G''$, whose mapping circles are known and given as the list $L=\{l_i\}$.
We know $R'$ is adjacent to $R$ over the edge $e$ of $G''$ and $e$ is in the image of the edge $E$ in $T''$.
Pull from $L$ the circles $l_i\in L$ which have in their description at least one 2--cell containing $E$.
Deconstruct the circles $l_i$ into a single unordered list $K$ of 2--cells.
Call the list of 2--cells in $T$ containing $E$ by $\sigma(E)$.
Note that in $\sigma(E)$ is a 2--cell created by an edge blowup that is projected over $R$ but not $R'$, and that 2--cell is absent from $K$.
The list of 2--cells that project over $R$ includes all of the piecewise linear circles in $L\setminus K$, as well as the symmetric difference of $K$ and $\sigma(E)$.
Recreating the circles in $L\setminus K$ is unnecessary as they are unchanged by passing over $e$.
We've gone through great pains in our explicit description of the dge blow up in order to ensure that the piecewise--linear circles in $K\bigtriangleup L$ are realized using the exact algorithm in Section~\ref{alg:trilist}.
More precisely, the chasing of 2--cells over 3--cells is unambiguous.
We now have a list of piecewise--linear circles in $T''$ for any region of $G''$.






