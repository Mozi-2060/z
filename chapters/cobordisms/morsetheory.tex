\label{sec:morsetheory}

Our primary use of Morse theory is to define a pair of related constructions.
The first is the \emph{handle decomposition} of a manifold, which comes with a dual decomposition.
The second is the \emph{Stein complex} of a Morse function on a manifold.
The main results from this chapter connect a Stein complex with a handle decomposition, and this construction is extended to triangulations in the remaining chapters.

\begin{defn}
	A generic smooth function $f:M\to\R$ where $M$ is an $n$--manifold is called a \emph{Morse function}.
	For a critical point $p\in M$ of $f$, the \emph{index} of $p$ is the dimension of the largest subspace of $T_p M$ such that $d^2f_p$ is negative definite.
	Less formally, this corresponds to the number of independent directions in $M$ along which $f$ decreases.
	We denote by $M^a$ the subspace $f\inv(-\infty,a\,]$ where $f$ is a fixed Morse function on $M$.
\end{defn}

A key question of Morse theory regards how the topology of $M^a$ changes as $a$ passes the critical values of $f$.
This question is answered by the following two theorems.

\begin{theorem}
	\label{thm:morseretract}
	Let $f:M\to\R$ be a Morse function with no critical values in $(a,b\,]$, $a<b$.
	If $f\inv[\,a,b\,]$ is compact, then $M^a$ and $M^b$ are diffeomorphic, and $M^b$ deformation retracts onto $M^a$.
\end{theorem}

\begin{theorem}
	\label{thm:morsehandle}
	Let $f:M\to\R$ be a Morse function.
	Let $p$ be a critical point of $f$ of index $\lambda$ with associated critical value $f(p)=q$.
	If $f\inv[\,q-\varepsilon, q+\varepsilon\,]$ is compact and contains no critical points other than $p$, then $M^{q+\varepsilon}$ is diffeomorphic to $M^{q-\varepsilon}\cup_\varphi H^\lambda$ for some attaching map $\varphi:S^{\lambda-1}\times\D^\mu\to \pd(M^{q-\varepsilon})$.
\end{theorem}

\begin{defn}
	Let $f:M\to\R$ be a Morse function on a closed $n$--manifold $M$ with critical points $\{p_1,\dots,p_k\}$ of indices $\{\lambda_1,\dots,\lambda_k\}$ and a set $\{t_0,\dots,t_k\}$ of numbers in $\R$ such that
	\[
		0\leq t_0 < f(p_1) < t_1 < f(p_2) < \cdots < t_{k-1} < f(p_k) < t_k \leq 1,
	\]
	and
	\[
		\lambda_1 \geq \lambda_2 \geq \cdots \geq \lambda_k.
	\]
	The previous theorems guarantee that, for each $i\geq 1$, $f\inv[\,t_{i-1},t_i\,]$ is diffeomorphic to $(f\inv(t_{i-1})\times\I)\cup H^{\lambda_i}$, giving us a realization of $M$ as
	\[
		\emptyset = M_0 \subset M_1 \subset M_2 \subset \cdots \subset M_{k-1} \subset M_k = M
	\]
	where $M_i$ is obtained from $M_{i-1}$ by attaching a $\lambda_i$--handle.
	Such a realization is called a \emph{handle decomposition} of $M$.
\end{defn}

A Morse function $f$ satisfying the requirements to define a handle decomposition actually defines two handle decompositions --- one from $f$ and one from $1-f$.
The handle decompositions from $f$ and $1-f$ are dual in the sense that every $\lambda$--handle attached in the making of the decomposition from $f$ has a corresponding $\mu$--handle in the decomposition from $1-f$.
This handle is realized by reversing the roles of the core and co--core in a given $\lambda$--handle $H^\lambda$ to obtain a $\mu$--handle $H^\mu$.

Similar results can be found for manifolds with boundary.
Let $W$ be an $n$--manifold with boundary $\pd W = M_0\sqcup M_1$, where both $M_0$ and $M_1$ are compact.
The Morse functions on $W$ that extend Theorems \ref{thm:morseretract} and \ref{thm:morsehandle} are those that fix $f(M_0)=0$, $f(M_1)=1$.
Now, the handle decomposition of $W$ is built on top of the component $M_1\times\{1\}$ of $M_0\times\I$, with $M_0\times\{0\}$ corresponding to the boundary component $M_0$ in the finished product.




\begin{defn}
	\label{def:steinfactorization}
\end{defn}

\begin{defn}
	\label{def:steincomplex}
\end{defn}



Let $f$ be a Morse function or Morse 2--function with compact fibers as before.
Define the \emph{Stein factorization} of $f$ to be a factorization $f=g\comp h$ so that the following are satisfied:
\begin{enumerate}
	\item The map $h$ has connected fibers, and
	\item the map $g$ is finite--to--one.
\end{enumerate}
We can also see the image of $h$ as the quotient space of the domain of $f$ by the connected fibers of $f$.
The image of $h$ will also be called the \emph{Stein complex} of the map $f$.

In the case of the oriented surface $\Sigma$, the Stein complex of $f$ is a graph whose vertices correspond to the singularities of $f$.
Vertices corresponding to critical points of index 0 or 2 being of degree 1, and vertices corresponding to critical points of index 1 being of degree 3.
We form a 3--manifold whose boundary is $\Sigma$ by starting with a disc bundle over the edges of $h(\Sigma)$, then extending over the vertices by attaching 3--balls as in the analysis above.
There is an obvious deformation retraction of the resulting 3--manifold to the Stein complex of $f$.

In the case of the 3--manifold $M$, the Stein complex is generically a surface and the possible singularities are more interesting.
As in the analysis above, some singularities are just the singularities from the lower dimensional Morse function crossed with an interval, so the corresponding shape in the Stein complex is simply a degree--one or degree--three vertex crossed with an interval.
The more interesting parts of the Stein complex are brought forth at the crossing of indefinite folds in the plane.
As in the case of $\Sigma$, $M$ is a circle bundle over the Stein complex at generic points and the 4--manifold $W$ with $\pd W = M$ is a disc bundle.
We extend over singularities by attaching handles and, as before, $W$ retracts onto the Stein complex.
The Stein complex, along with some extra information we can obtain from $f$, provides a set of building instructions for $W$ explored in the next section.


\begin{cor}
	Let $M$ be a closed, orientable 3--manifold.
	Then there exists a Morse 2--function $f:M\to\RR$ with Stein factorization $f=g\comp h$ and a Stein complex $S=h(M)$ whose regions can be assigned framing constants such that $S$ provides a full set of instructions for the construction of a 4--manifold $W$ with $\pd W=M$.
\end{cor}





\begin{defn}
	\label{def:morse2function}
	A generic smooth function $f:M\to\RR$ where $M$ is an $n$--manifold is called a \emph{Morse 2--function}.
\end{defn}







In a similar fashion to when we worked with Morse functions, assume that $M$ is closed and oriented and require that the image of $f$ is contained inside of $\DD\subset\RR$.
From the regular value theorem, the preimage of a regular value of $f$ is the disjoint union of oriented circles in $M$.
The image of the singular set of $f$ is a collection of arcs in $\RR$ that only intersect pairwise and transversally.
This leaves us with a classification of critical values of $f$ into codimension 1 singularities, i.e.\ arcs away from crossings, and codimension 2 singularities, i.e.\ arc crossings.

