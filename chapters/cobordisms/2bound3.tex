\label{sec:2bound3}

We now have the necessary framework to demonstrate that a closed, oriented 2--manifold bounds a 3--manifold through an explicit construction.
This is useful as a jumping off point for the case one dimension higher, that of 3--manifolds bounding 4--manifolds, which draws on the same framework and methods.

Let $\Sigma$ be an oriented surface and $f:\Sigma\to\R$ a Morse function with distinct critical values.
We begin by examining the preimages of points $x\in\R$, denoting the space $f\inv(x)$ by $\Sigma_x$.
A space $\Sigma_x$ may not be connected, so we index the connected components by superscript.
By the regular value theorem, any regular value pulls back through $f\inv$ to a disjoint collection of oriented circles in $\Sigma$.
Let $p$ be a critical point of $f$ with critical value $f(p)=x$.
By our assumptions, $p$ is the only critical point of $f$ in $\Sigma_x$.
The connected component of $\Sigma_x$ containing $p$ is called the \emph{critical level} at $p$ and is denoted $\Sigma_x^p$.
The remaining connected components of $\Sigma_x$ are all copies of $S^1$, and we will index them by $\Sigma_x^i$, for $i=1,\dots,k$.

consists of the union of a connected subspace of $\Sigma$ containing $p$, called the critical level of $f$ at $p$ and denoted $\Sigma_x^p$, with a disjoint collection of circles.

The shape of the critical level is determined by the index of $p$.
Because $f$ is a Morse function whose domain is a surface, its critical points are easily classified by the Morse lemma, Lemma \ref{lem:morselemma}.
Locally in $\Sigma$, a critical point of index 0 is a minimum, of index 1 a saddle, and of index 2 a maximum of $f$.

Let $\varepsilon(x)=(x-\varepsilon,x+\varepsilon)$ be an interval containing $x$ and no other critical values of $f$.
Denote the neighbourhood $f\inv(\varepsilon(x))$ of $\Sigma_x$ by $\varepsilon(\Sigma_x)$, and denote the connected component of $\varepsilon(\Sigma_x)$ containing $p$ by $\varepsilon^p(\Sigma_x)$.
When $p$ is of index 0 or 2, we can immediately deduce that $\varepsilon^p(\Sigma_x)$ is diffeomorphic to a disc.
When $p$ is of index 1, the Morse lemma tells us that $\Sigma$ looks like a standard saddle near $p$.
The intersection of $\Sigma_x^p$ with this saddle is a cross whose centre is $p$.
For $y\in\varepsilon(x)$, $y$ is a regular value whose preimage is a disjoint union of oriented circles.
The circles above and below the saddle singularity are the result of smoothing out the cross into a pair of oriented arcs, done in two possible ways.
The orientations of these circles orient the cross, which has two incoming arms and two outgoing arms which appear in alternating order.
A Morse function has distinct critical levels, so the cross we know about in $\Sigma_x^p$ must have its arms connected in $\Sigma_x^p$ through nonsingular orientation--preserving arcs.
We can then see that $\Sigma_x^p$ is a figure 8, and $\varepsilon^p(\Sigma_x)$ is a is a pair of pants in $\Sigma$.

The components of $\varepsilon(\Sigma_x)$ that contain no critical points of $f$ are disjoint unions of annuli, one for each circle in $\Sigma_x$.
Similarly, the preimage of an interval of regular values in $\R$ is the disjoint union of annuli in $\Sigma$.

\{figure: smoothing a cross into a saddle\}

\{figure: neighbourhoods of critical points (index 0,1,2)\}

Let $\Sigma\times I$ be a 3--manifold with oriented boundary components $\Sigma\times\{0\}=\Sigma_0$ and $\Sigma\times\{1\}=\Sigma_1$, where $\Sigma_0$ has orientation opposite that of $\Sigma$ and $\Sigma_1$.
Let $f$ be a Morse function on $\Sigma_0$ satisfying the assumptions of Definiton \ref{def:morsehandle}.
Take $f=g\comp h$, where $g$ is the map that crushes the connected components of $\Sigma_x$ to distinct points, and $h$ is the map that 


easy away from the critical values of $f$.

 Specify a regular value of $f$ in each connected component of $f(\Sigma_0)$ minus the critical values of $f$.
Each of these regular values has preimage a disjoint collection of circles in $\Sigma_-$ which we take to be the attaching spheres of 3--dimensional 2--handles. 
The resulting object is a 3--manifold with one boundary component of $\Sigma_+$ and the rest corresponding to singular values of $f$.
To obtain a 3--manifold whose boundary is $\Sigma_+$, we need only fill the boundary components corresponding to the singular values of $f$.

\{FIGURE: MORSE FUNCTION AND 2--HANDLE ATTACHMENT\}



We return now to the 3--manifold constructed from $\Sigma\times I$ by gluing 2--handles to the $\Sigma_-$ boundary component.
The boundary components consist of $\Sigma_+$ as well as a collection of components corresponding to the critical points of $f$ left over in the $\Sigma_-$ side of the boundary.
If $p$ has index 0 or 2, then the component of $f\inv(\varepsilon(x))$ contianing $p$ is a disc.
For some $\epsilon$, the boundary of this disc meets the belt sphere of the nearest 2--handle, and the boundary component is seen to be a 2--sphere, which becomes the attaching sphere of a 3--handle.
Similarly, if $p$ has index 1 then the pair of pants near the saddle singularity of $p$ is grown until it meets three belt spheres from three separate 2--handle attachments.
The boundary component is again seen to be a 2--sphere, so we can attach another 3--handle.
It is reasonable to note that, had $\Sigma$ not been oriented, then the boundary component corresponding to a saddle singularity could have been a copy of $\RP^2$.
As $\RP^2$ has Euler characteristic 1, it does not bound any 3--manifold.

\{FIGURE: THREE TYPES OF 3--BALL BLOCKS\}
