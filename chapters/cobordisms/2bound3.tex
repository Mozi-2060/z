\label{sec:2bound3}

We now have the necessary framework to demonstrate that a closed, oriented 2--manifold bounds a 3--manifold through an explicit construction.
This is useful as a jumping off point for the case one dimension higher, that of 3--manifolds bounding 4--manifolds, which draws on the same framework and methods.
To demonstrate, we let $f:M\to\RR$ be a Morse 2--function from the 3--manifold $M$.	
From the regular value theorem, the preimage of a regular value of $f$ is the disjoint union of oriented circles in $M$.	
The image of the singular set of $f$ is a collection of arcs in $\RR$ that only intersect pairwise and transversally.
This leaves us with a classification of critical values of $f$ into codimension 1 singularities, i.e.\ arcs away from crossings, and codimension 2 singularities, i.e.\ arc crossings.
We obtain a 4--manifold with boundary $M$ by gluing discs to regular preimage circles and extending over each singularity type.
	
To get an idea of how to extend over these singularities, we examine what happens one dimension down.
A closed 2--manifold $\Sigma$ bounds a 3--manifold if and only if the Euler characteristic $\chi(\Sigma)$ is even.
In particular, every orientable surface bounds a 3--manifold.
Let $\Sigma$ be an oriented 2--manifold and $f$ a Morse function $\Sigma\to\R$.
Preimages of regular values are again circles, which we fill with discs.
We use the language of handles to describe this filling.
Let $\Sigma\times I$ be a 3--manifold with oriented boundary components $\Sigma\times\{0\}=\Sigma_-$ and $\Sigma\times\{1\}=\Sigma_+$, where $\Sigma_-$ has orientation opposite that of $\Sigma$ and $\Sigma_+$.
Specify a regular value of $f$ in each connected component of $f(\Sigma_-)$ minus the critical values of $f$.
Each of these regular values has preimage a disjoint collection of circles in $\Sigma_-$ which we take to be the attaching spheres of 3--dimensional 2--handles. 
The resulting object is a 3--manifold with one boundary component of $\Sigma_+$ and the rest corresponding to singular values of $f$.
To obtain a 3--manifold whose boundary is $\Sigma_+$, we need only fill the boundary components corresponding to the singular values of $f$.

\{FIGURE: MORSE FUNCTION AND 2--HANDLE ATTACHMENT\}

Because $f$ is a Morse function whose domain is a surface, its critical points are classified.
Let $p$ be a critical point of $f$, and $x$ its critical level.
Locally in $\Sigma_-$, a critical point of index 0 is a minimum, of index 1 a saddle, and of index 2 a maximum of $f$.
Let $\varepsilon(x)=(x-\varepsilon,x+\varepsilon)$ be an interval containing $x$ and no other critical values of $f$.
When $p$ is of index 0 or 2, we can immediately deduce that the component of $f\inv(\varepsilon(x))$ containing $p$ is a disc in $\Sigma_-$.
To see this, recognize that $f\inv(x)$ will consist of a disjoint collection of circles corresponding to the regular points of $\Sigma_-$ that also map to $x$ through $f$ along with a single point -- $p$.

\{FIGURE: NEIGHBOURHOODS OF INDEX 0,1,2 CRIT. PTs\}

The case of the saddle needs more care.
Suppose that $p$ is of index 1.
Critical points of Morse functions are non--degenerate, so $f\inv (x)$ forms a cross near $p$.
For $y\in\varepsilon(x)$ not equal to $x$, $y$ is a regular value whose preimage is a disjoint union of circles.
As $\Sigma_-$ is oriented, pulling back the orientation of $\R$ to $\Sigma_-$ allows us to coherently orient the preimage circles of the regular values.
The circles above and below the saddle singularity are the result of smoothing out the cross into a pair of oriented arcs, done in two possible ways.
The orientations of these circles orient the cross, which has two incoming arms and two outgoing arms which appear in alternating order.
A Morse function has distinct critical levels, so the cross we know about in the connected component of $f\inv(x_0)$ containing $p_0$ must have its arms connected in $\Sigma_-$ through nonsingular orientation--preserving arcs.
The connected component of $f\inv(x_0)$ containing $p_0$ must then be a figure 8.
Thus, the component of $f\inv(\varepsilon)$ containing $p$ is a pair of pants in $\Sigma_-$.

\{FIGURE: SMOOTHING OUT A CROSS AND SADDLE\}
