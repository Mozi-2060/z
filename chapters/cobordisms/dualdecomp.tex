	To satisfy our requirement that the handle decomposition contains handles of index at most 2, we turn the decomposition obtained on its head and take its dual.
	This method begins with an empty space and attaches 4--dimensional handles to it until the boundary of the handle decomposition is exactly $M$.
	
	Let $W_0$ be an empty decomposition.
	The cocores of our 4--handles are single points, so each 4--handle yields a 0--handle in the dual decomposition.
	Add a disjoint 4--disc (i.e.\ a 0--handle) to $W_0$ for each 4--handle added in the final step of building the original decomposition.
	We have after this $W_1=\bigsqcup \D^4$ whose boundary is $\pd W_1 = \bigsqcup S^3$.
	This is the same boundary that we had before adding any 4--handles to the original decomposition.
	
	The cocores of our 3--handles are intervals that connect 4--handles.
	The belt sphere of a 3--handle is the boundary of the cocore, so it is a copy of $S^0$ intersecting exactly two 4--handles.
	We connect the associated 0--handles by 3--disc bundles over copies of the interval whose endpoints are embedded in our 0--handles (i.e.\ attach 1--handles) in a way that preserves orientability.
	We get $W_2$ as the 4--dimensional thickening of a graph whose vertices correspond to 0--handles and whose edges correspond to 1--handles.
	The boundary of this is the collection of $S^3$'s from $\pd W_1$ that have been connected together by copies of $S^2\times\I$.

	Attaching 0--handles and 1--handles in a way that preserves orientability is done uniquely.
	This is not the case with 4--dimensional 2--handles (see Remark \ref{rmk:2handle}).
	The 2--handles must be attached in a way that is dual to the attachment described in the original handle decomposition, else the boundary of the resulting 4--manifold need not be $M$.
	
	An original 2--handle is attached over a solid torus $V$ with $f(V)=D$, and the desired action on the boundary $M_0$ was the replacement of $V$ by $V^*$, a solid torus that could be filled in by a 3--handle.
	The core of the handle attached over $V$ is the disc bounded by the zero section $z(V)$, and the cocore of the handle is the disc bounded by the zero section $z(V^*)$.
	The core and cocore intersect at $\vec{0}$, and nowhere else.
	
	The attachment of 3--and 4--handles filled in $V^*$, so adding 0-- and 1--handles to our dual decomposition recovers $V^*$, the 4--manifold constructed contains $V^*$, running through $S^3$'s and along $S^2\times\I$'s, but not $V$.
	We give $V^*$ the structure of a trivial 2--disc bundle over the belt sphere of the original 2--handle with trivialization $\psi:V^*\to S^1\times\DD$.
	In the boundary of $V^*$ we find the curves $J$, $K$ and $L$ as defined before, displayed in Figure \ref{fig:VV*}.
	Just as it was when we were attaching the original 2--handle, Figure \ref{fig:VV*} is slightly misleading.
	What it really shows us is how the curves sit in $\psi(V^*)$.
	
	In $S^1\times\DD$, $\psi(J)$ is a meridian, $\psi(L)$ is a curve that wraps once around $\pd(S^1\times\DD)$ in the meridinal direction and $\kappa$ times around $\pd(S^1\times\DD)$ in the longitudinal direction, with oriented intersection count with $\psi(J)$ of $\kappa$, and $\psi(K)$ is exactly $S^1\times\{1\}$ in $\pd(S^1\times\DD)$.
	
	For a 2--handle attached over $V^*$, take $\pd (\DD\times\DD)$ to be the genus 1 Heegaard splitting of $S^3$ as before.
	One of the solid tori will be identified with $V^*$ and the other will be $V^{**}$.
	We need an explicit identification $G^*:V^*\to S^1\times\DD$ such that $G^*(K)$ is a meridian of $V^{**}$
	
	At this point it becomes convenient to view $T^2$ as the space $[0,1]\times[0,1]/\sim$, where $(\theta,0)\sim(\theta,1)$ and $(0,\phi)\sim(1,\phi)$.
	Using the notation from Theorem \ref{thm:mpgV}, a meridian is of the form $(\theta,0)$, and $\psi(J)$ is then isotopic to a line with slope $1/0$, and $\psi(K)$ is isotopic to the longitude $(0,\phi)$, which is a line with slope $0/1$.
	We find that $\psi(L)$ is isotopic to a line of slope $-1/\kappa$, or, alternatively, is a line of the form $-\kappa\phi=\theta$.