At its simplest, most colloquial definition, an $n$--dimensional manifold is a space that locally looks like real $n$--dimensional space or half space $\HN$.
We use \emph{charts} and \emph{atlases} to make explicit what is meant by ``looks like.''

\begin{defn}[Coordinates]
	\label{def:coordinates}
	Let $\HN\subset\RRN$ denote the closed real half space under the subspace topology, defined as
	\[
		\HN=\{(x_1,\dots,x_{n})\in\RRN : x_n\geq 0\}.
	\]
	We use the notations $\inter{\HN}$ and $\pd \HN$ to denote the interior and boundary of $\HN$ as subsets of $\RRN$ which, when $n>0$, are
	\[
		\begin{array}{ccccc}
			\inter{\HN} & = & \{(x_1,\dots,x_{n})\in\RRN : x_n > 0\}, & &\\
			\pd \HN	& = & \{(x_1,\dots,x_{n})\in\RRN : x_n = 0\} & \approx & \R^{n-1}.
		\end{array}
	\]
	When $n=0$, $\HH^0 = \R^0 = \{0\}$, so $\inter{\HH^0}=\R^0$ and $\pd\HH^0=\emptyset$.

	Let $X$ be a second-countable Hausdorff space.
	The pair $(U,f)$ where $U$ is an open subset of $X$ and $f$ is a homeomorphism from $U$ onto an open set of either $\RRN$ or $\HN$ is called a \emph{chart} of $X$.
	The map $f$ is a \emph{coordinate system} on $U$ and its inverse $f\inv$ is a \emph{parameterization} of $U$.
	Writing $f$ as
	\[
		f(u) = (\xi_1(u),\dots,\xi_n(u)),
	\]
	the functions $\xi_i$ are \emph{coordinate functions}
\end{defn}

\begin{defn}[Atlas]
	\label{def:atlas}
	Let $\mathcal{A}=\{(U_\alpha,f_\alpha)\}$ be a collection of charts of the space $X$ parameterized by $\alpha\in A$ for some indexing set $A$.
	If $\bigcup_A U_\alpha$ contains $X$, then $\mathcal{A}$ is an \emph{atlas} for $X$.
	The homeomorphisms $f_\alpha\comp f_\beta\inv:f_\beta(U_\alpha\cap U_\beta)\to f_\alpha(U_\alpha\cap U_\beta)$ are \emph{transition maps} of $\mathcal{A}$.
	We say that $(U_\alpha,f_\alpha)$ and $(U_\beta,f_\beta)$ are \emph{smoothly compatible} if either $U_\alpha\cap U_\beta$ is empty or the transition maps $f_\alpha\comp f_\beta\inv$ and $f_\beta\comp f_\alpha\inv$ are smooth as maps $\RRN\to\RRN$, i.e.\ they have continuous partial derivatives of all orders.
	If every pair of charts in an atlas is smoothly compatible then that atlas is a \emph{smooth atlas}.
	Two smooth atlases are equivalent if their union is a smooth atlas.
\end{defn}

\begin{defn}[Manifolds]
	\label{def:manifold}
	Let $X$ be a second-countable Hausdorff topological space, and $\mathcal{A}$ an atlas for $X$.
	If $\mathcal{A}$ is a smooth atlas, then the pair $(X,\mathcal{A})$ is a \emph{smooth n--manifold},  \emph{n--manifold}, or just \emph{manifold}.
	We usually omit writing the atlas when talking about a manifold.
	If $\mathcal{A}$ is a maximal smooth atlas then we call it a \emph{smooth structure} on $X$.
	We will assume that a smooth manifold is always equipped with a smooth structure.
	If $X$ is an $n$--manifold and $Y\subset X$ satisfies the definition of an $m$--manifold under the subspace topology, then $Y$ is an $m$--dimensional \emph{submanifold} of $X$.
\end{defn}

\begin{defn}[Boundary, Interior]	
	\label{def:boundary}
	Let $X$ be a smooth $n$--manifold.
	A chart $(U,f)$ is called an \emph{interior chart} is $f(U)$ is an open subset of $\RRN$ and is a \emph{boundary chart} if $f(U)$ is an open subset of $\HN$ with $f(U)\cap\pd \HN\neq\emptyset$.
	Let $p\in X$.
	We say that $p$ is an \emph{interior point} if it is in the domain of an interior chart, and is a \emph{boundary point} if it is in the domain of a boundary chart $(U,f)$ so that $f(p)\in\pd\HN$.
	The set of all boundary points of $X$ is called the \emph{boundary of $X$} and is denoted by $\pd X$.
	Similarly, the set of all interior points of $X$ is called the \emph{interior} of $X$ and is denoted by $\inter{X}$.
	If $X$ is compact with empty boundary then $X$ is \emph{closed} as a manifold.
\end{defn}

\begin{prop}[Boundaries are Manifolds]
	\label{prop:boundariesaremanifolds}
	Let $X$ be an $(n+1)$--manifold.
	The boundary $\pd X$ is an $n$--dimensional submanifold of $X$ with empty boundary.
\end{prop}

\begin{proof}
	Let $\pd X\subset X$ be a second-countable Hausdorff space under the subspace topology.
	Let $\mathcal{A} = \{(U_\alpha, f_\alpha)\}$ be a smooth structure on $X$.
	There is a smooth atlas $\mathcal{B}$ for $\pd X$ that can be described in term of $\mathcal{A}$.
	For each $(U_\alpha,f_\alpha)$ in $\mathcal{A}$, there is a corresponding chart $(U_\beta,f_\beta)$ in $\mathcal{B}$ where $U_\beta = U_\alpha \cap \pd X$ and $f_\beta = \restr{f_\alpha}{U_\beta}$.
	Each $f_\beta$ is a homeomorphism whose domain is a set $U_\beta$ that is open in $\pd X$.
	Each point in $U_\beta$ is a boundary point and $f_\beta$ is a homeomorphism, so $f_\beta(U_\beta)$ is an open subset of $\pd \HH^{n+1} \approx\RRN$.
	The union of $U_\alpha$'s contains $X$, so the union of $U_\beta$'s contains $\pd X$, so $(\pd X, \mathcal{B})$ is a smooth $n$--manifold.
	Any chart $(U,f)$ of $\mathcal{B}$ has $f(U)$ an open subset of $\RRN$, so $(U,f)$ is an interior chart and $\pd(\pd X)$ is therefore empty.
	We conclude that $\pd X$ is a smooth closed $n$--manifold. 
\end{proof}

Some important results that help build our theory use as a tool smooth maps between manifolds.
In particular, smooth maps allow us to declare a notion of equivalence between manifolds.

\begin{defn}[Smooth Map]
	\label{def:smoothmap}
	Let $(X,\{U_\alpha,f_\alpha\})$ and $(Y,\{V_\beta,g_\beta\})$ be smooth $n$-- and $k$--manifolds respectively and let $\varphi:X\to Y$ be a map between them.
	If, for any $\alpha$ and $\beta$, the composition $g_\beta\comp \varphi\comp f_\alpha\inv$ is smooth as a map $\RRN\to\R^k$, then we say $\varphi$ is \emph{smooth} as a map between manifolds.
	If $\varphi$ is smooth and a well-defined $\varphi\inv$ exists and is smooth, then $\varphi$ is called a \emph{diffeomorphism} between manifolds.
	We say that manifolds are \emph{diffeomorphic} if there exists a diffeomorphism between them.
\end{defn}

\begin{prop}
	\label{prop:diffeoequiv}
	Diffeomorphism is an equivalence relation on the space of smooth manifolds.
\end{prop}

Another important concept in the study of manifolds is that of the tangent space.
Manifolds are defined by their local homogeneity, and a tangent space is a precise description of that homogeneity near a point.

\begin{defn}[Tangent Space]
	\label{def:tangentspace}
	Let $X$ be an $n$--manifold, $p$ an interior point of $X$, $(U,f)$ a chart containing $p$ with $f(p)=\vec{0}$, $B_r^k$ the open ball of radius $r$ centred at $\vec{0}$ in $\R^k$, and $\gamma$ a map $$\gamma:B_r^k\to \inter{X}$$
	with $\gamma(0)=p$.
	We say that $\gamma$ is \emph{smooth} in a neighbourhood of $t$ in $B_r^k$ if $f\comp\gamma$ is smooth at $t$ as a map $\R^k\supset B_r^k\to\RRN$.
	When $k=1$ and $r$ is some small $\varepsilon$, $B_r^k$ is the interval $(-\varepsilon, \varepsilon)$.
	In this case, when $\gamma$ is smooth on the whole of the interval $(-\varepsilon, \varepsilon)$, such a $\gamma$ is called a \emph{curve} in $X$ through $p$.
	Note that this definition is independent of the chart used, as all of our transition maps are smooth.
	
	Let $C_p X$ be the space of smooth curves in $X$ through $p$ and $\gamma_1, \gamma_2$ elements of $C_p X$.
	We can define an equivalence relation $\sim$ on $C_p(X)$ by saying that $\gamma_1\sim\gamma_2$ if
	\[
		\Ddt(f\comp\gamma_1)(0) = \Ddt(f\comp\gamma_2)(0).
	\]
	An equivalence class of the curve $\gamma$ in $C_p X$ is a \emph{tangent vector} at $p$ and is written as $\gamma'(0)$.
	The space $C_p X/\sim$ is the \emph{tangent space} at $p$, denoted $T_p X$.	
\end{defn}

\begin{prop}
	\label{prop:tangentspacevectorspace}
	Let $X$ be an $n$--manifold and $p$ a point in $\inter{X}$.
	Then $T_p X$ is a vector space isomorphic to $\RRN$.
\end{prop}

\begin{proof}
	Let $(U,f)$ be a chart containing $p$ with $f(p)=\vec{0}$.
	It follows from our definition of tangent space that the map defined by
	\[
		\begin{array}{crcl}
			df : & T_p X & \to & \RRN \\
			& \gamma'(0) & \mapsto & \Ddt(f\comp \gamma)(0) .
		\end{array}
	\]
	is a bijection, so we also have a well defined inverse $(df)\inv$.
	We define operations on $T_p X$ so that $F$ is strengthened to a vector space isomorphism:
	\[
		\begin{array}{rcl}
			\gamma_1'(0)+\gamma_2'(0) & = & (df)\inv(df(\gamma_1'(0))+df(\gamma_2'(0))), \\
			t\gamma_1'(0) & = & (df)\inv(t\:df(\gamma_1'(0))).
		\end{array}
	\]
	The vector space structure of $T_p X$ is independent of the choice of chart.
	To see this, let $L_v$ be the parameterized straight line through $\vec{0}$ whose velocity is $v\in\RRN$.
	Precisely, $L_v(t)=tv.$
	For any chart $(U,f)$ with $f(p)=0$, the curve $f\inv\comp L_v$ is a curve through $p$ for which $df(f\inv\comp L_v)=v$.
	Because an equivalence class contains such a curve for any applicable chart, the association of a tangent vector with a vector in $\RRN$ does not depend on the chart used.
\end{proof}

In normal calculus, the derivative is a linearization of a function.
We use tangent spaces to define the derivative of a smooth map between manifolds because tangent spaces are used as pointwise linearizations of manifolds.

\begin{defn}[Differential, Pushforward]
	Let $\varphi:X\to Y$ be a smooth map between manifolds.
	Let $p\in X$ and $\varphi(p)=q\in Y$.
	The \emph{differential} of $\varphi$ at $p$ is a linear map defined as
	\[
		\begin{array}{crcl}
			d\varphi: & T_p X & \to & T_q Y\\
					  & \gamma'(0) & \mapsto & (\varphi\comp\gamma)'(0),
		\end{array}
	\]
	and the element $(\varphi\comp\gamma)'(0)$ of $T_q Y$ is the \emph{pushforward} of the tangent vector $\gamma'(0)$.
\end{defn}

We also classify smooth maps between manifolds.
This classification contributes to the idea of an \emph{embedded submanifold} and provides some of the groundwork for defining handles in the next chapter.

\begin{defn}[Submersion, Immersion, Embedding]
	Let $\varphi:X\to Y$ be a smooth map between manifolds.
	The \emph{rank} of $\varphi$ at the point $p\in X$ is the rank of the differential $d\varphi$.
	This is computed as the rank of the Jacobian matrix of $\varphi$ under a coordinate system or as the dimension of $\varphi(T_p X)\subset T_q Y$.
	If $\varphi$ has rank $k$ for every $p$ in $X$, then $\varphi$ is of \emph{constant rank} and we say $\rk\varphi=k$.
	
	If $d\varphi$ is injective at each point, then $\varphi$ is an \emph{immersion}.
	This is equivalent to $\rk{\varphi}=\dimn X$.
	If $d\varphi$ is surjective at each point, then $\varphi$ is a \emph{submersion}.
	This is equivalent to $\rk{\varphi}=\dimn Y$.
	If $\varphi$ is an injective immersion that is a homeomorphism onto its image $\varphi(X)\subset Y$ under the subspace topology, then it is a \emph{smooth embedding} or just \emph{embedding}.
\end{defn}



The main result of this work is applicable to orientable 3--manifolds, so we will review what is meant by a space being orientable.
Essentially, orientability is a guarantee that when you go for a walk your right and left sides haven't switched places by the time you get home.

\begin{defn}[Orientation]
	\label{def:orientation}
	Let $\RRN$ be $n$--dimensional real space, $\mathcal{B}(\RRN)$ the set of ordered bases for $\RRN$, and $\gl{n}{\R}$ the general linear group, i.e.\ the space of $n\times n$ invertible matrices with entries in $\R$.
	Let $b_1$ and $b_2$ be any two ordered bases from $\mathcal{B}(\RRN)$.
	It is a standard result of linear algebra that there is a unique element $A$ of $\gl{n}{\R}$ that transforms $b_1$ into $b_2$.
	If the determinant $\det A$ is positive, then $b_1$ and $b_2$ are \emph{positively oriented} with respect to each other.
	If $\det A$ is negative, then $b_1$ and $b_2$ are \emph{negatively oriented}.
	We can define an equivalence relation $\sim$ on $\mathcal{B}(\RRN)$ by saying that $b_1\sim b_2$ if they are positively oriented.
	An \emph{orientation} of $\RRN$ is a choice of one of the two equivalence classes of $\mathcal{B}(\RRN)$.
	This also allows us to classify linear transformations of $\gl{n}{\R}$ by their action on the quotient space $\mathcal{B}(\RRN)/\sim$ by saying they are \emph{orientation preserving} if they have positive determinant and \emph{orientation reversing} if they have negative determinant.
	
	Suppose we have fixed an orientation of $\RRN$.
	Let $X$ be an $n$--manifold.
	We say that an \emph{orientation} of $X$ is a consistent choice of orientation of the tangent space at every point of $X$.
	A consistent choice of orientation means that for every chart $(U,f)$ with $f:U\to\RRN$, the vector space isomorphism $df:T_p X\to \RRN$ is orientation preserving at every point in $U$. 
	If $X$ admits an orientation, then it is \emph{orientable}.
	If $X$ does not admit an orientation, then it is \emph{non-orientable}.
\end{defn}

A common tool in both the construction and description of a manifold is the \emph{bundle}.

\begin{defn}[Bundles]
	A \emph{fibre bundle} is most concisely represented by the composition
	\[
		F\into E\overset{\pi}{\onto} B,
	\]
	where $B$ is called the \emph{base space}, $E$ the \emph{total space}, $F$ the \emph{fibre}, and $p$ the \emph{projection}.
	For the tuple $\beta = (F,E,B,\pi)$ to be a fibre bundle, we require that $\pi$ be continuous, that the subspaces $F_x = \pi\inv(x)$ are each homeomorphic to $F$, and that for every point $x\in B$ there exists a neighbourhood $U$ of $x$ and a homeomorphism
	\[
		\varphi: U\times F\to \pi\inv(U)
	\]
	so that $\pi\comp\varphi(y,v)=u$ for any pair $(y,v)$ in $U\times F$.
	Any point $p\in E$ may be represented uniquely by a pair $p=(x,v)$ with $x\in B$, $v\in F$, and $\pi(p)=x$.
	When $F$ is a vector space, we call $\beta$ a \emph{vector bundle}.
	A function $s:B\into E$ so that $s(x)\in F_x$ for every $x\in B$ is called a \emph{section}.
	When $\beta$ is a vector bundle, we can define the \emph{zero section} of $\beta$ to be the map
	\[
		\begin{array}{cccc}
			z: & B & \to 	 & E \\
			   & x & \mapsto & (x,0).
		\end{array}
	\]
	We also call the subspace $z(B)\subset E$ the zero section of $E$.
\end{defn}

\begin{defn}
	Let $\Phi:\beta_0\to \beta_1$ be a map between vector bundles.
	We call $\Phi$ a \emph{fibre map} if $\Phi:E_0\to E_1$ covers a map $\varphi:B_0\to B_1$.
	For $\Phi$ to cover $\varphi$, that means the following diagram commutes:
	\[
		\begin{tikzcd}
			E_0 \arrow{r}{\Phi} \arrow[swap]{d}{\pi_0} & E_1 \arrow{d}{\pi_1} \\
			B_0 \arrow{r}{\varphi} & B_1
		\end{tikzcd}
	\]
	This means that if $x\in B_0$ and $\varphi(x)=y$, then $\Phi$ maps $F_x$, the fibre over $x$, to $F_y$, the fibre over $y$ by a map we will denote $\Phi_x$.
	If $\Phi_x$ is a linear map for every $x\in B_0$, then we call $\Phi$ a \emph{bundle morphism}.
	If $\Phi$ is a bundle morphism, $B_0=B_1=B$, $\Phi_x$ bijective for each $x\in B$, and $\varphi$ is $\ident{B}$, then $\Phi$ is a bundle \emph{isomorphism}.
\end{defn}

\begin{defn}[Tangent Bundle]
	Let $X$ be an $n$--manifold.
	We define the total space $TX$ of a vector bundle with base space $X$ and fibres $\RRN\approx T_p X$ at any $p\in X$ by taking the disjoint union of all tangent spaces:
	\[
		TX = \bigsqcup_{p\in X} T_p X.
	\]
	The projection $\pi:TX\to X$ is defined by $\pi(q)=p$ for every $q\in T_p X$.
	We call $TX$ the \emph{tangent bundle} over $X$.
\end{defn}

\begin{defn}[Normal Bundle]
	Let $X$ be a $k$--dimensional submanifold of the $n$--manifold $Y$.
	At a point $p\in X\subset Y$, the tangent space $T_p X$ is a subspace of the tangent space $T_p Y$.
	Denote the orthogonal complement to $T_p X$ in $T_p Y$ by $N_p X$ and call it the normal space at $p$ in $X$.
	That is, $T_p X\oplus N_p X = T_p Y$.
	We define the total space $NX$ of a vector bundle with base space $X$ and fibres $\RRN\approx T_p X$ at any $p\in X$ by taking the disjoint union of all normal spaces:
	\[
		NX = \bigsqcup_{p\in X} N_p X.
	\]
	The projection $\pi:NX\to X$ is defined by $\pi(q)=p$ for every $q\in N_p X$.
	We call $NX$ the \emph{normal bundle} over $X$.	
\end{defn}

\begin{defn}[Tubular Neighbourhood]
	\label{def:tubularneighbourhood}
	Let $X\subset Y$ be a submanifold with $\pd X = \pd Y = \emptyset$.
	An embedding $f:NX\to Y$ with $f(x,0)=x$ and $f(NX)$ an open neighbourhood of $X$ in $Y$ is called a \emph{tubular neighbourhood} of $X$ in $Y$.
	Denote the neighbourhood as a subset of $Y$ by $\nu(X)=f(NX)$.
\end{defn}

\begin{theorem}[Tubular Neighbourhood, Theorem 4.5.2 in \cite{Hirsch67}]
	\label{thm:tubularneighbourhood}
	Let $X$ be a smooth closed submanifold of the smooth closed manifold $Y$.
	Then $X$ has a tubular neighbourhood in $Y$.
\end{theorem}

Tubular neighbourhoods are unique up to a fibre-preserving isotopy, so lets define precisely what that means.

\begin{defn}[Homotopy]
	\label{def:homotopy}
	Let $f,g:X\to Y$ be smooth maps between smooth manifolds.
	Denote the closed unit interval $[0,1]$ by $\I$.
	A function 
	\[
		\begin{array}{cccc}
			H: & X\times \I & \to & Y \\
			   & (x,t)	& \mapsto & H_t(x)
		\end{array}
	\]
	with $H_0(x)=f(x)$ and $H_1(x)=g(x)$ is a \emph{homotopy} between $f$ and $g$.
	If a homotopy exists, then $f$ and $g$ are \emph{homotopic}.
	Less formally, $f$ and $g$ being homotopic means that one can be continuously deformed into the other.
	The topological spaces $X$ and $Y$ are \emph{homotopy equivalent} if there exist continuous maps $f:X\to Y$ and $g:Y\to X$ for which $g\comp f$ is homotopic to $\ident{X}$ and $f\comp g$ is homotopic to $\ident{Y}$.
	
	Because we are primarily interested in smooth functions to build our machinery, we extend our definition of homotopy to a smooth version.
	With the notation above, a smooth map $H:X\times[0,1]\to Y$ with $H_0(x)=f(x)$ and $H_1(x)=g(x)$ is a \emph{smooth homotopy} between $f$ and $g$.
	If a smooth homotopy exists, then $f$ and $g$ are \emph{smoothly homotopic}.
\end{defn}

A homotopy through embeddings is called an isotopy.
This is useful for comparing embedded submanifolds.

\begin{defn}[Isotopy]
	\label{def:isotopy}
	Let $X$ be a smoothly embedded submanifold of $Y$.
	An \emph{isotopy} of $X$ in $Y$ is a smooth homotopy
	\[
		\begin{array}{crcl}
			F: & X\times \I & \to & Y \\
			   & F(x,t) & = & F_t(x)
		\end{array}
	\]
	such that the related map
	\[
		\begin{array}{cccc}
			\hat{F}: & X\times\I & \to & Y\times\I \\
					 & (x,t) & \mapsto & (F_t(x),t)
		\end{array}
	\]
	is an embedding.
	The submanifolds $F_0(X)$ and $F_1(X)$ are \emph{isotopic}.
	When $X=Y$ and $F_t$ is a diffeomorphism for each $t$, $F$ is a \emph{diffeotopy} of $Y$.
	
	Let $X$ be a smoothly embedded submanifold of $Y$ and consider the tubular neighbourhood $f:NX\to Y$.
	An isotopy
	\[
		\begin{array}{crcl}
			F: & \nu(X)\times\I & \to & Y \\
			   & F(x,t) & = & F_t(x)
		\end{array}
	\]
	satisfying the following properties:
	\begin{enumerate}
		\item $F_0(\nu(X))=F_1(\nu(X))$,
		\item $F_1\inv\comp F_0:\nu(X)\to\nu(X)$ is a vector bundle isomorphism $NX\to NX$,
	\end{enumerate}
	 is an \emph{isotopy of tubular neighbourhoods}.
\end{defn}

This leads into our uniqueness result for tubular neighbourhoods.

\begin{prop}[Uniqueness of Tubular Neighbourhoods, Theorem 4.5.3 in \cite{Hirsch67}]
	\label{prop:uniquenesstubularneighbourhood}
	Let $X\subset Y$ be a submanifold with $\pd X=\pd Y=\emptyset$.
	Any pair of tubular neighbourhoods of $X$ in $Y$ are isotopic.
\end{prop}	

\begin{theorem}[Isotopy Extension]
	\label{thm:isotopyextension}
	Let $X$ be a smooth compact submanifold of the smooth closed manifold $Y$, and let
	\[
		\begin{array}{crcl}
			F: & X\times\I & \to & Y \\
			   & F(x,t) & = & F_t(x)
		\end{array}
	\]
	be an isotopy of $X$ in $Y$.
	Let $L$ be the subset of $Y$ equal to the union of the images of $F_t$ for each $t$.
	More precisely,
	\[
		L = \bigcup_{t\in[0,1]} F_t(X).
	\]
	There exists a diffeotopy 
	\[
		\begin{array}{crcl}
			G: & Y\times\I & \to & Y \\
			   & G(y,t) & = & G_t(y)				
		\end{array}
	\]
	with $G_0=\ident{Y}$, $G_1$ equal to $F_1$ on $X\subset Y$, and $F_t$ the identity on $Y$ outside of an arbitrarily small neighbourhood of $L$ for all $t$.
\end{theorem}

\begin{defn}[Ambient Isotopy]
	\label{def:ambientisotopy}
	Let $X$, $Y$, $F$ be as above.
	The isotopy $G:Y\times\I\to Y$ guaranteed by Theorem \ref{thm:isotopyextension} is called an \emph{ambient isotopy}.
	The images $F_0(X)$ and $F_1(X)$ are \emph{ambiently isotopic} as submanifolds of $Y$.
\end{defn}
