

\begin{defn}[Coordinates]
	\label{def:coordinates}
	Let $X$ be a topological space and let $U\subset X$ and $V\subset \RRN$ be open.
	A homeomorphism $f:U\to V$ is called a \emph{coordinate system} on $U$.
	Writing $f$ as $f(u)=(x_1(u),\dots,x_n(u))$, the functions $x_i$ are \emph{coordinate functions}.
	The pair $(f,U)$ is a \emph{chart} of $X$, and the inverse map $f\inv$ is called a \emph{parameterization} of $U$.
\end{defn}

\begin{defn}[Atlas]
	\label{def:atlas}
	Let $\mathcal{A}=\{U_\alpha,f_\alpha\}$ be a collection of charts of $X$ parameterized by $\alpha\in A$.
	If $\bugcup_A U_\alpha$ contains $X$, then $\mathcal{A}$ is an \emph{Atlas} for $X$.
	The homeomorphisms $f_\alpha\comp f_\beta\inv:f_\beta(U_\alpha\cap U_\beta)\to f_\alpha(U_\alpha\cap U_\beta)$ are \emph{transition maps}.
	
\end{defn}


\begin{defn}[Manifolds]
	\label{def:manifold}
	A separable Hausdorff topological space $X$ is an $n$--dimensional \emph{topological manifold} if, for every point $p$ in $X$, there is a neighbourhood $U$ of $p$ and a homeomorphism $f:V\to U$ where $V$ is an open subset of the $n$--dimensional real half space $$\RNP=\{(x_1,\dots,x_{n})\in\RRN : x_n\geq 0\}.$$
	The pair $(U,f)$ is a \emph{chart}, and a collection of charts $\{(U_\alpha,f_\alpha):\alpha\in A\}$ such that the union $\bigcup_A U_\alpha$ contains $X$ is an \emph{atlas}.
	The map $f_\alpha\inv\comp f_\beta$ defined on $f_\beta\inv(U_\alpha\cap U_\beta)$ is called the \emph{transition map} between the charts $(U_\alpha,f_\alpha)$ and $(U_\beta,f_\beta)$.
	If $\{(U_\alpha,f_\alpha):\alpha\in A\}$ is an atlas of $X$ and the transition maps $f_\alpha\inv\comp f_\beta$ are smooth maps between subsets of $\RNP$ for every $\alpha,$ $\beta$ pair in $A$, then we call $X$ a \emph{smooth} manifold.
	If the transition maps are piecewise--linear, then $X$ is a \emph{piecewise--linear} manifold.
	When we would like to emphasize the dimension, we say that $X$ is an $n$--dimensional manifold or an $n$--manifold.
	If $X$ is an $n$--manifold and $Y\subset X$ satisfies the definition of an $m$--manifold, then $Y$ is an $m$--dimensional \emph{submanifold} of $X$.
\end{defn}

With that taken care of, we can stop talking about topological manifolds.
All future instances of manifold objects and statements about them will carry the assumption that the manifold in question is either smooth or PL (piecewise--linear) unless stated otherwise.
We will revise this assumption to make it stricter when we come to Chapter \ref{cha:alg1}, but it will suffice until then.

\begin{defn}[Boundary, Interior]	
	\label{def:boundary}
	Let $X$ be a manifold.
	A point $x$ of $X$ covered by chart $(U,f)$ with $f\inv(x)$ in $\{(x_1,\dots,x_n): x_n = 0\}$ is called a \emph{boundary point} of $X$.
	The set of all boundary points of $X$, called the \emph{boundary of $X$}, is denoted by $\pd X$.
	A point $x$ in $X\setminus \pd X$ is called \emph{interior}.
	If $X$ is compact and $\pd X=\emptyset$, then $X$ is said to be a \emph{closed manifold}.
\end{defn}

\begin{prop}[Boundaries are Manifolds]
	\label{prop:boundariesaremanifolds}
	Let $X$ be a manifold.
	The boundary of $X$, $\pd X$, is an $(n-1)$--dimensional closed submanifold of $X$.
\end{prop}

\begin{proof}
	content...
\end{proof}

Brief Digression On Orientations of $\RRN$.

\begin{defn}
	\label{def:orient}
	
	here we define tangent spaces and orientability of a manifold
	
\end{defn}

\begin{defn}
	\label{def:isotopy}
	
	A smooth \emph{isotopy} between embeddings $\varphi_0,\varphi_1:Y\to X$ is a smooth homotopy $\varphi_t: Y \to X$ ($0\leq t\leq 1$) through embeddings.
	If an isotopy exists, we say that $\varphi_0,\varphi_1$ are \emph{isotopic}.
	
\end{defn}

brief digression on isotopy extension

\begin{theorem}[Isotopy Extension \{CITE: Miln\}]
	\label{thm:isotopyextension}
	
	Let $M$ be a smooth compact submanifold of the smooth closed manifold $N$.
	If $h_t$ is a smooth isotopy of the inclusion $M\into N$ then $h_t$ is the restriction of a smooth isotopy $h_t':N\to N$ of the identity $\ident{N}$ such that $h_t'$ is the identity on $N$ outside of a compact subset of $N$.
	
\end{theorem}

We would like to restate the above theorem into a more useful form.
Let $Y\subset X$ be a smooth compact submanifold in the interior of the smooth manifold $X$ and let $i:Y\into X$ be inclusion.
Any smooth isotopy $\varphi_t:Y\to X$ with $\varphi_0=i$ induces an isotopy $\Phi_t:X\to X$ through diffeomorphisms $X\to X$ with $\Phi_0=\ident{X}$ and $\varphi_t=\Phi_t\comp\varphi_0$ for each $t$.
The isotopy $\Phi_t$ is called an \emph{ambient isotopy}, and a pair of submanifolds $Y_1,Y_2$ of $X$ are \emph{ambiently isotopic} is there exists a diffeomorphism $\Phi:X\to X$ such that $\Phi$ is homotopic to $\ident{X}$ through diffeomorphisms and $\Phi$ maps $Y_1$ to $Y_2$.
