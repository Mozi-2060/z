The \emph{CW--complex} is our first foray into combinatorial descriptions of manifolds.

\begin{defn}
	For $n>0$, a topological space that is homeomorphic to the $n$--ball is called an \emph{open $n$--cell}, and a topological space that is homeomorphic to the $n$--disc is called a \emph{closed $n$--cell}.
	For $n=0$, we say that a singleton set is both a closed and open 0--cell.
	
	Let $M$ be a Hausdorff space.
	A \emph{cell decomposition} of $M$ is a collection $T=\{t_\alpha\}$ such that every $t_\alpha$ is an open cell, every pair of cells of $T$ is disjoint, and the union of the cells of $T$ is exactly $M$.
	The \emph{$n$--skeleton} of $T$ is the disjoint union of the cells of $T$ with dimension at most $n$, and it is denoted $T^n$.
	Note that the 1--skeleton of a cell decomposition is a graph.
	
	Let $T$ be a cell decomposition of the Hausdorff space $M$.
	Then $T$ is a \emph{closure--finite weak--topology complex} if the following are satisfied: 
	\begin{enumerate}
		\item For any $n$--cell $t_\alpha$ in $T$, there exists a continuous map $f:\DN\to M$ such that $\restr{f}{B^n}$ is a homeomorphism onto $t_\alpha$ and $f(S^{n-1})$ is contained in $T^{n-1}$.
		\item The closure of any cell of $T$ intersects only a finite number of other cells in $T$ (i.e.\ $T$ is closure finite).
		\item A subset $X$ of $M$ is closed if and only if $X\cap \overline{t_\alpha}$ is closed for every $t_\alpha$ in $T$ (i.e. the topology of $M$ is weak).		
	\end{enumerate}
	It is standard to abbreviate such a cell decomposition as a \emph{CW--complex}.	
	The \emph{dimension} of a CW--complex $T$ is the highest dimension among the cells of $T$.
	A CW--complex is \emph{edge distinct} if no pair of edges $e$, $f$ in $T^1$ have $\pd e=(u,v)=\pd f$.
	A map $\phi:T\to X$ where $X$ is an $m$--manifold and $T$ is a $n$--complex is an \emph{embedding} if $\phi$ restricted to any cell of $T$ is an embedding.
	That embedding is \emph{flat} if for every point $\phi(p)$ of any top dimensional embedded $n$--cell $\phi(c)$ of $T$, there is a chart $(U,f)$ of $X$ about $\phi(p)$ for which $f(U\cap X)$ sits inside of $\R^n\subset \R^m$.
\end{defn}

We concern ourselves mostly with a specific type of cell decomposition called a \emph{triangulation.}
The cells of a triangulation are the interiors of \emph{$n$--simplices}, which can be thought of as generalizations of the most basic 2--dimensional building block, the triangle, to arbitrary dimensions.

\begin{defn}
	Let $E=\{e_0,\dots,e_n\}$ be a set of $n+1$ points in some $n$--dimensional affine space $F$ such that
	\[
	  e_i\neq \sum_{j\neq i}t_j e_j
	\]
	for each $i$ and any collection of nonnegative real $t_j$ with $\sum t_j=1$.
	The \emph{standard} $n$--\emph{simplex} $\sigma$ is defined as
	\[
	  \sigma = \{p\in F : p = \sum_{i=1}^n t_i e_i,\text{ } e_i\in E,\text{ } t_i\geq 0\text{ for each }i, \text{ and }\sum_{i=1}^n t_i=1\}.
	\]
	The above construction is called taking the \emph{convex hull} over $E$.
	
	Let $E'$ be a $k+1$ element subset of $E$.
	The convex hull over $E'$ is called a \emph{facet} or \emph{$k$--facet} of $\sigma$ and is itself a $k$--simplex.
	A zero--dimensional facet is a \emph{vertex}, one--dimensional an \emph{edge}, two--dimensional a \emph{triangle}, three--dimensional a \emph{tetrahedron} and four--dimensional a \emph{pentachoron}.
	These names are also applied to the 0--, 1--, 2--, 3--, and 4--simplices.
	A facet with dimension $n-1$ is called a \emph{face} of the simplex that contains it.
	We number the vertices of the $n$--simplex with the numbers $0,\dots,n$.
	Every face of $\sigma$ contains all but one vertex of $\sigma$, and this gives a numbering to the faces of $\sigma$.
	The $i^{th}$ face of $\sigma$ is found via the \emph{face map} $F^i(\sigma)$.
	Removing all of the proper faces of a simplex leaves us with the \emph{interior} of $\sigma$.
\end{defn}

Let's start building some cell structures out of simplices.
First, notice that a simplex has a canonical cell decomposition into the interiors of its facets.
Next, let $\sigma$, $\tau$ be $n$--simplices, $n>0$.
A linear attaching map $F^i(\sigma)\to F^j(\tau)$ can be defined by mapping the vertices of $F^i(\sigma)$ to the vertices of $F^j(\tau)$ and then extending linearly over the facets of $F^i(\sigma)$.
Let $g:F^i(\sigma)\to F^j(\tau)$ be such an attaching map, and form the space $\sigma\cup_g \tau$.
There is also a canonical cell decomposition of $\sigma\cup_g \tau$ built from the interiors of the facets of $\sigma$ and $\tau$.
Extending to a collection of $n$--simplices and gluing maps, a cell decomposition continues to be simple to define, and we may say the adjunction formed from simplices in a collection $\{\sigma_i\}$ over gluing maps $\{g_j\}$ has a canonical cell decomposition.

\begin{defn}
  A CW--complex matching the description above is called an \emph{$n$--dimensional simplicial gluing complex} or just \emph{gluing complex}.
  The faces of a gluing complex $T$ can be partitioned into two subsets based on whether the face has been glued to another face.
  If a face $F^i(\sigma)$ has been glued to another face $F^j(\tau)$, both faces are \emph{glued}.
  Otherwise, $F^i(\sigma)$ is \emph{unglued}.
  The set of unglued faces and all facets contained in those faces form the \emph{boundary} of a gluing complex, and is denoted $\pd T$.
  The remaining facets are \emph{internal}.
  A gluing complex with empty boundary is \emph{closed}.

  Put $n=k+m$ and let $\sigma$ be an $m$--simplex in an $n$--gluing complex $T$ and consider an $n$--simplex $\tau$ containing $\sigma$.
  There is a $k$--facet $\tau_k$ that is ``opposite'' $\sigma$ in the sense that $\sigma\cap\tau_k=\emptyset$ and $\tau$ is the convex hull of $\sigma\cup\tau_k$.
  The subset of $C$ built as the union of all $\tau_k$ corresponding to $\tau$ containing $\sigma$ is the \emph{link} of $\sigma$ and is denoted $\lk{\sigma}$.
  Note that $\lk{\sigma}$ is an $m$--gluing complex in its own right.

  If $M$ is an $n$--manifold and $T$ is an $n$--simplicial gluing complex that is homeomorphic to $M$, then we say that $T$ is a \emph{triangulation} of $M$.
  A triangulation $T$ with the property that any pair of distinct edges share at most one vertex is called \emph{edge distinct}.
\end{defn}

There is a very slick way of checking whether a gluing complex is a triangulation of some $n$--manifold using only the links of vertices.

\begin{theorem}
  An $n$--gluing complex $T$ is the triangulation of some $n$--manifold $M$ if and only if the link of every internal vertex $v$ is homeomorphic to $S^{n-1}$ and the link of every boundary vertex $w$ is homeomorphic to $D^{n-1}$ with $\pd\lk{w}$ contained in $\pd T$.
\end{theorem}

\begin{defn}
  When we have a triangulation $T$ for a closed, oriented $n$--manifold $M$, we can also define a structure dual to that triangulation that is useful for computation.
  This structure, denoted $T^*$, is a cell decomposition dual to $T$ in the sense that each $(n-k)$--cell of $T^*$ is associated with a $k$--cell of $T$.
  For a $k$--simplex $\sigma$ of $T$, denote the dual cell in $T^*$ to $\sigma$ by $\sigma^*$.
  When $T^*$ is fully defined, this forms a bijection between the $k$--cells of $T$ and the $(n-k)$--cells of $T^*$.
  Now, $\sigma$ is contained in a finite number of $n$--simplices $\tau_i$, and we can fully describe $\sigma^*$ by describing how $\sigma^*$ meets these $\tau_i$.

  For a given $\tau$ containing $\sigma$, there are $2^{n-k}$ subsets of the vertices of $\tau$, i.e.\ the 0--skeleton of $\tau$, i.e.\ $\tau^0$, that contain the vertices of $\sigma$.
  Each subset $s_j\subset\tau^0$ has a barycentre $b(s_j)$, so we form a collection $B_\tau(\sigma)=\{b(s_j)\}$ of these barycentres.
  Then the intersection of $\sigma^*$ with $\tau$ is defined as the convex hull over $B_\tau(\sigma)$ inside of $\tau$.
  The union of these hulls over all $\tau_i$ containing $\sigma$ is then a closed $(n-k)$--cell whose interior is then defined to be the dual cell $\sigma^*$.
  The union of all $\sigma^*$ forms a CW--complex $T^*$ for $M$ that we call the \emph{dual decomposition} to $T$.
  
  When $T$ is a triangulated 3--manifold, the top dimensional dual cells of $T^*$ are all solid polyhedra.
  We call a 3--dimensional CW--complex whose top--dimensional cells are all polyhedra a \emph{solid polyhedral gluing}.
  Note that a 3--manifold triangulation is also a solid polyhedral gluing.
\end{defn}

A description of a manifold through a cell decomposition offers us a finite, combinatorial object that we are able to tinker with and investigate using methods that are computable, and the algorithms found in future chapters all use cell decompositions as their main data type.
The next logical question is whether cell decompositions accurately represent smooth manifolds.
Fortunately, we are working in dimension at most 4.
This means that a triangulation has an essentially unique smoothing, and triangulations of smooth manifolds exist and are well defined.
The standard references for these facts are \cite{HirsMazu} with respect to unique smoothings, and \cite{Whit40} with respect to the existence of triangulations.
