A simplicial manifold is a special description of a piecewise--linear structure on a piecewise--linear manifold.
We describe it as being built from a patchwork of pieces carved from a finite dimensional Euclidean space which we call simplices.

Let $E=\{e_0,\dots,e_n\}$ be a set of $n+1$ points in some $n$--dimensional affine space $F$ such that
\[
  e_i\neq \sum_{j\neq i}t_j e_j
\]
for each $i$ and any collection of nonnegative real $t_j$ with $\sum t_j=1$.
A \emph{standard} $n$--\emph{simplex} $\sigma$ is defined as
\[
  \sigma = \{p\in F : p = \sum_{i=1}^n t_i e_i , e_i\in E, t_i\geq 0\text{ for each }i,\text{ and }\sum_{i=1}^n t_i=1\}.
\]
The above set is also called the \emph{convex hull} of $E$.

Let $E'$ be a $k+1$ element subset of $E$.
The convex hull on $E'$ is called a \emph{facet} of $\sigma$ and is itself a $k$--simplex.
A zero--dimensional facet is a \emph{vertex}, one--dimensional an \emph{edge}, two--dimensional a \emph{triangle}, three--dimensional a \emph{tetrahedron} and four--dimensional a \emph{pentachoron}.
These names are also applied to the 0--, 1--, 2--, 3--, and 4--simplices.
A facet with dimension $n-1$ is called a \emph{face}.
We number the vertices of the $n$--simplex with the numbers $0,\dots,n$
Every face of $\sigma$ contains all but one vertex of $\sigma$, and this gives a numbering to the faces of $\sigma$.
The $i^{th}$ face of $\sigma$ is found via the \emph{face map} $F^i(\sigma)$.
Removing from a simplex all of its proper faces gives us the \emph{interior} of $\sigma$.

\{FIGURE HERE IS GOOD\}

We define a \emph{simplicial complex}, or just \emph{complex}, to be a locally finite collection $\Sigma$ of simplices embedded in some affine space satisfying two conditions.
First, any facet of a simplex in $\Sigma$ is also in $\Sigma$.
Second, the intersection of two simplices in $\Sigma$ is either empty or a facet of both.

An $n$--\emph{dimensional gluing} consists of a finite set of $n$--simplices, and a collection of face pairs $(F^i(\sigma),F^j(\tau))$ and gluing maps $g$.
We demand that any face appears in exactly one of the face pairs, and $g$ is an affine linear map that dictates how the faces $F^i(\sigma)$ and $F^j(\tau)$ are identified.
The gluing of the faces $F^i(\sigma),F^j(\tau)$ is defined by a bijection of the vertices of $F^i(\sigma)$ to the vertices of $F^j(\tau)$ and extended linearly in order of ascending dimension over all facets of the faces.
We are interested in the quotient space of the union of our simplices by the equivalence relation defined by our gluing maps, and this quotient space is often referred to as a \emph{gluing}.
An $n$--dimensional gluing can be seen as a simplicial complex by embedding it in some high--dimensional Euclidean space.
Denote an $n$--gluing by $T$ and define the \emph{$k$--skeleton} of $T$ to be the union of all facets of $T$ of dimension at most $k$ with $0\leq k\leq n$.
Denote the $k$--skeleton of $T$ by $T^k$, and note that $T^n=T$.

Recall from Definition \ref{def:manifold} that an $n$--dimensional piecewise--linear manifold $M$ is a manifold whose atlas of $M$ has piecewise linear transition maps.
If a gluing satisfies this property, we call it a \emph{simplicial manifold}.
To build an atlas for an $n$--gluing $T$, a point $p$ in $T\setminus T^{n-1}$ has an obvious chart $(\inter{\sigma},f)$ where $\sigma$ is the $n$--simplex whose interior contains $p$ and $f$ is the trivial linear map from $\inter{\sigma}$ to the interior of the standard $n$--simplex, an open subset of $\RRN$.
Because a chart for every point interior to an $n$--simplex exists and because no face of this complex is unglued, we may iteratively define piecewise--linear homeomorphisms for points in $T\setminus T^{k}$ as $k$ descends to zero.
This leaves the task of defining charts for the vertices of $T$.
An important notion in constructing these charts is the \emph{link} of a simplex.

\begin{defn}
  If $\sigma$ is a $k$--simplex in an $n$--gluing $T$, consider a simplex $\tau$ containing $\sigma$.  
  Then there is a facet $\tau'$ that is ``opposite'' $\sigma$ in the sense that $\sigma\cap \tau'=\emptyset$ and $\tau$ is the convex hull of $\sigma\cup \tau'$.
  The gluing made up of the $F^i(\tau)$ corresponding to all $\tau$ containing $\sigma$ is the \emph{link} of $\sigma$ and is denoted $\lk{\sigma}$.
\end{defn}

If the link of a vertex $v$ in $T$ has $\lk{v}=S^{n-1}$, then the cone on that link is a neighbourhood of $v$ in the gluing and is piecewise-linearly homeomorphic to the $n$--disc.
Because the rest of the atlas is immediate from our definition of gluing, the condition that vertex links are spheres is necessary and sufficient to say that a gluing is a simplicial manifold.

Until now, our definitions only allow for closed simplicial manifolds.
If we allow our $n$--dimensional gluing to have unpaired faces, then we impose the additional restriction that the links of unglued vertices are $(n-1)$--dimensional discs with boundary $(n-2)$--spheres triangulated by facets entirely from unglued faces.
If this additional condition is met, then the unglued faces form an $(n-1)$--dimensional simplicial manifold in their own right.

A theorem of Whitehead \{CITE: Whitehead 1940\} tells us that every smooth manifold has a canonical piecewise--linear structure hence a structure as a simplicial manifold.
It is also true that every simplicial manifold of dimension $\leq 6$ has an essentially unique smoothing up to diffeomorphism \{CITE:Milnor, topology 46 years later\}
