The main tool used to construct 4--manifolds in later chapters is handle attachment.
We focus mainly on the definitions and results needed to meaningfully attach 2--handles to a 4--manifold.

\begin{defn}
\label{def:handle}
Let $X$ be a smooth $n$--manifold and let $0\leq k\leq n$.
We use $D^k$ to denote the closed $k$--disc.
An \emph{$n$--dimensional $k$-handle}, denoted here by $h$, is a copy of $D^k\times D^{n-k}$ which we attach to $\pd X$ along $\pd D^k\times D^{n-k}$ by an embedding
\[
  \varphi : \pd D^k \times D^{n-k} \to \pd X.
\]
We call $\varphi$ an \emph{attaching map}, and canonically smooth corners \{CITE:GoSt\} in the quotient space
\[
  X\cup_\varphi h = \quotient{X \cup ( D^k\times D^{n-k} ) }{ \varphi(x) \sim x }
\]
to make $X\cup_\varphi h$ into a smooth manifold.
The number $k$ is the \emph{index} of the handle $h$ and we call this process \emph{attaching a $k$-handle to $X$ with attachment map $\varphi$}.
An $n$--dimensional manifold described by a handle decomposition whose handles have index no greater than $1$ is called an \emph{$n$-handlebody}.
\end{defn}





If $X\cup_\varphi h$ is a smooth manifold with attached $k$-handle, then we have special terms for the anatomy of $h$.
The $k$--disc at $D^k\times 0$ in $h$ is called the \emph{core} of $h$ and the $(n-k)$--disc $0 \times D^{n-k}$ is called the \emph{cocore}.
We call the $(k-1)$--sphere $\pd D^k \times 0$ the \emph{attaching sphere} and we call the $(n-k-1)$--sphere $ 0 \times \pd D^{n-k}$ the \emph{belt sphere}.
we call $\pd D^k\times D^{n-k}$, or its image through an attaching map $\varphi$, the \emph{attaching neighbourhood}.

\begin{rmk}
If $\varphi_0$ and $\varphi_1$ are isotopic attaching maps of the $k$--handle $h$ then the manifolds $X\cup_\varphi h$ and $X\cup_{\varphi'} h$ are diffeomorphic up to ambient isotopy.
This follows from the Isotopy Extension Theorem.
If $\varphi_0,\varphi_1$ are isotopic maps into the closed manifold $\pd X$, then the restrictions of these maps to the compact attaching neighbourhood are also isotopic hence induce an ambient isotopy $\Phi_t$.
The diffeomorphism
\begin{eqnarray*}
  \ident{I}\times \Phi : I\times \pd X 
  & \to &
  I\times \pd X    \text{\hspace{10mm}defined by}  \\
  (t,x)
  & \mapsto &
  (t,\Phi_t(x))
\end{eqnarray*}
gives us the diffeomorphism $X\to X$ by identifying $I\times\pd X$ with a neighbourhood of $\pd X$ in $X$. \{CITE: RoSa\}

We construct an attaching map $\varphi:\pd D^k\times D^{n-k}\to\pd X$ from an embedding of the attaching sphere $\varphi_0:\pd D^k\times 0\to \pd X$ and an identification $n$ of the normal bundle of the image of $\varphi$ with $\pd D^k \times \R^{n-k}$.
From the Tubular Neighbourhood Theorem, this information determines an embedding of $\pd D^k\times D^{n-k}$ up to isotopy.
Thus any attaching map is determined by
\begin{enumerate}
  \item an embedding $\varphi_0:S^{k-1}\to\pd X$ with trivial normal bundle, and
  \item an identification $n$ of the normal bundle $\nu\varphi_0(S^{k-1})$ with $S^{k-1}\times\R^{n-k}$ which we call a \emph{framing of} $\varphi_0(S^{k-1})$
\end{enumerate}
up to isotopy, and the resulting manifold $X\cup_\varphi h$ up to diffeomorphism.
\end{rmk}

\begin{rmk}
\label{rmk:onehandle}
There are two important cases of handle attachment that we care about --- attachment of handles with index $1$ and $2$.
To attach a handle of index $1$ to an $(n+1)$--manifold $W$ over its $n$--manifold boundary $M$, we specify an embedding $\varphi_0$ of $S^0=\{-1,1\}$ into $M$.
We must also define a framing of $\varphi_0(S^{0})$ in $M$.
If $M$ is oriented, then a framing of $\varphi_0(\pm 1)$ is an element of $\gl{n}{\R}$.
Because an orientation of the $n$--manifold $M$ is defined by an association of an element of $\glp{n}{\R}$ to each point of $M$, we can say that a framing of $\varphi_0(\pm 1)$ is \emph{orientation preserving} if the chosen framing element is from $\glp{n}{\R}$ and \emph{orientation reversing} if the chosen framing element is from $\gln{n}{\R}$.
With these framings defined, we can form the attaching map $\varphi$ for our 1--handle $h$.
Because the attaching map $\varphi$ is defined on the disconnected space $S^0\times D^n$, we refer to $\varphi$ as a pair of attaching maps, one for each component. 
Note here that if $W$ is path connected and both framings are orientation reversing or orientation preserving, then the manifold $W\cup_\varphi h$ is nonorientable.
If exactly one framing is orientation preserving and the other orientation reversing then $W\cup_\varphi h$ is orientable.
\end{rmk}



Because our goal is to attach 2--handles to 4--manifolds, we focus now on embeddings of $\sone$ in 3--manifolds.
We can define a framing of an embedded $\sone$ in an oriented 3--manifold from a vector field over that embedded $\sone$ and the orientation of the 3--manifold.
Attaching a 2--handle to a 4--manifold consists of defining an embedding of $\sone\times D^2$ into the boundary of the 4--manifold.
This embedding is defined entirely by the image of the attaching sphere and by a framing of that image.
A framing is an association of every point of the attaching sphere with an element of $\gl{2}{\R}$.
We care about this embedding up to isotopy, so a framing can be reduced to an element of $\pi_1(\gl{2}{\R})=\pi_1(\sone)=\Z$.
A framing makes no sense without a zero framing as reference.
Once that is defined, any $n$--framing is also defined.
There is a canonical zero--framing of a knot in $\sthr$ found by the outward unit normal vector to any Seifert surface of that knot.
