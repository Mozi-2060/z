The main tool used to construct 4--manifolds in later chapters is handle attachment.
We focus mainly on the definitions and results needed to meaningfully attach 2--handles to a 4--manifold, using the methods that are explored in detail in \cite{Kosi93} to attach handles in way that ensures the resulting manifold is smooth.
This is accomplished by piping the areas to be joined together through cooked up diffeomorphisms.
We lay out the general theory behind joining manifolds along submanifolds of the boundary, then move into the specifics of handle attachment.

Let $\beta=(\R^k,E,Y,\pi)$ be a $k$-vector bundle with $Y$ a compact $n$--manifold.
Let $\alpha:(0,\infty)\to(0,\infty)$ be an orientation reversing diffeomorphism.
Define a map $\alpha_E: E\setminus z(Y) \to E\setminus z(Y)$ by
\begin{equation}
	\label{eqn:alpha}
	\alpha_E(y,v) = \Big(y,\alpha(\norm{v})\frac{v}{\norm{v}}\Big).
\end{equation}
Let $X_1$, $X_2$ be $(n+k+1)$--manifolds with boundary, and let $h_1$, $h_2$ be embeddings of $E$ in $\pd X_1$ and $\pd X_2$.
Restricting the $h_i$ to $z(Y)$, we get $h_i(z(Y))$ as a embedded submanifolds of the boundaries $\pd X_i$.
We may then view the $(E,h_i)$ as tubular neighbourhoods of $h(z(Y))$ in $\pd X_i$, and can extend the neighbourhoods to $X_i$ as in Definition \ref{def:halfneighbourhood}, using two copies of $\HH^1$.
This extends the fibre of $\beta$ from $\R^k$ to $\R^k\times \HH^1$.

Denote the extended embeddings by $H_i$.
If we extend the bundle even further to $\mathcal{B}=(\R^k\times \R,\mathcal{E},Y,\pi)$, we can think of $H_1$, $H_2$ as each embedding ``one-half'' of $\mathcal{E}$ in the boundaries of $\pd X_1$ and $\pd X_2$.
We obtain a new object by ``joining'' $X_1$ and $X_2$ along the embeddings $H_1$, $H_2$.
Explicitly, identify $(y,v)$ in $H_1(\mathcal{E}\setminus z(Y))$ with $H_2\comp \alpha_\mathcal{E}\comp H_1\inv(y,v)$.
A space obtained in this manner will be denoted $X(h_1,h_2)$.

\begin{theorem}
	\label{thm:joinman}
	Let $\beta$, $X_i$, $h_i$ be as above.
	The space $X(h_1,h_2)$ is a smooth $(n+k+1)$--manifold unique up to bundle automorphisms of $E$.	
\end{theorem}

The machinery used to prove Theorem \ref{thm:joinman} is built up throughout Chapter \rom{6} of \cite{Kosi93}.
Now, we build up the details of handle attachment -- an explicit application of Theorem \ref{thm:joinman}.

Let $n=\lambda+\mu$ and $X$ be an $n$--manifold with boundary.
Attaching an $n$--dimensional $\lambda$--handle to $X$ is the process of joining $X$ to $D^n$ along embeddings of $\beta=(\R^\mu,E,S^{\lambda-1},\pi)$, a $\mu$--vector bundle over $S^{\lambda-1}$, into the boundaries of $X$ and $D^n$.
%An explicit embedding of $E$ into $\pd D^n$ is explored in depth in \rom{6}.6 of \cite{Kosi93}, so we will touch only on the facts necessary to explicitly define handles.

Write elements of $\RRN=\R^\lambda\times\R^\mu$ as $x=(x_\lambda,x_\mu)$, where $x_\lambda$ is the result of projecting $x$ onto the first $\lambda$ coordinates of $\RRN$ and $x_\mu$ is the result of projection onto the last $\mu$ coordinates.
This notation lets us write $S^{\mu-1}$ in $\RRN$ as
\[
	S^{\mu-1} = \{x\in D^n: x_\lambda=\vec{0},\norm{x_\mu}=1\}.
\]
For $\varepsilon\in [\,0,1)$ define $T(\varepsilon)$ as a subset of $D^n$ by
\[
	T(\varepsilon) = \{x\in D^n: \norm{x_\lambda}^2>\varepsilon\}.
\]
Note that $T(\varepsilon)$ collapses onto $S^{\lambda-1}$ via the projection $(x_\lambda,x_\mu)\mapsto x_\lambda/\norm{x_\lambda}$.
Abbreviate $T(0)$ as $T$.
Define a map
\[
	\alpha:T(\varepsilon)\setminus S^{\lambda-1} \to T(\varepsilon)\setminus S^{\lambda-1}
\]
defined by
\begin{equation}
	\alpha(x_\lambda, x_\mu) =
	\Bigg( 
		\frac{x_\lambda}{\norm{x_\lambda}}
		(1-\norm{x_\lambda}^2+\varepsilon)^{1/2},
		x_\mu
		\frac{
			(\norm{x_\lambda}^2-\varepsilon)^{1/2}
		}{
			(1-\norm{x_\lambda}^2)^{1/2}
		}
	\Bigg).
\end{equation}
To relate handle attachment with our initial definition of joining manifolds over submanifolds in the boundary, first notice that $T(\varepsilon)$ has the structure of ``one-half'' of a $(\mu+1)$--vector bundle over $S^{\lambda-1}$.
Next, $\alpha$ is a composition of the diffeomorphism $D^n\setminus S^{\lambda-1}\to\inter{D^\lambda}\times D^\mu$ given by
\begin{equation}
	\label{eqn:diffeo}
	(x_\lambda,x_\mu)\mapsto \Big(x_\lambda,\frac{x_\mu}{(1-\norm{x_\lambda}^2)^{1/2}}\Big)
\end{equation}
with an involution on $(\inter{D^\lambda}\setminus\{\vec{0}\})\times D^\mu$ given by
\begin{equation}
	(x_\lambda,x_\mu)\mapsto \Big(\frac{x_\lambda}{\norm{x_\lambda}}(1-\norm{x_\lambda}^2+\varepsilon)^{1/2},x_\mu \Big)
\end{equation}
and then with the inverse of equation \ref{eqn:diffeo}.
Thus, $\alpha$ is an orientation reversing diffeomorphism as in equation \ref{eqn:alpha}.

\begin{defn}[Handle]
	Let $X$ be a smooth $n$--manifold with nonempty boundary and let $n=\lambda+\mu$.
	Let $h:S^{\lambda-1}\to \pd X$ be an embedding with extension $H:T\to X$ such that $H(T)$ is a tubular neighbourhood of $h(S^{\lambda-1})$ in $X$ in the sense of Definition \ref{def:halfneighbourhood}.
	We obtain a new space by identifying $x\in T\setminus h(S^{\lambda-1})$ with $H\comp\alpha(x)\in X\setminus h(S^{\lambda-1})$, a process we call \emph{attaching an $n$--dimensional $\lambda$--handle to $X$ along $h(S^{\lambda-1})$}.
	We denote the new space by $X\cup_h H^\lambda$.	
\end{defn}

Attaching a handle is a special case of joining a pair of manifolds along submanifolds of the boundary so Theorem \ref{thm:joinman} applies.
Thus, $X\cup_h H^\lambda$ is a smooth $n$--dimensional manifold unique up to automorphisms of $\mu$--vector bundles over $S^{\lambda-1}$.
This means that the characteristics of handle attachment that fully describe the manifold $X\cup_h H^\lambda$ are:
\begin{enumerate}
	\item The embedding $h:S^{\lambda-1}\to\pd X$ and its resulting tubular neighbourhood in $X$ which is unique up to ambient isotopy of $X$, and
	\item The structure of
	\item
	
\end{enumerate}

There is some specific language that is useful to the description of handle attachment,.
The subset of $X$ along which the handle is attached is $h(S^{\lambda-1})$


\newpage

\begin{center}
	What follows is unedited
\end{center}

\newpage

\begin{defn}[Handle]
	\label{def:handle}	
	Let $X$ be a smooth $n$--manifold with nonempty boundary and let $n=\lambda+\mu$.
	Let $h:S^{\lambda-1}\to \pd X$ be an embedding with extension $H:T\to X$ such that $H(T)$ is a tubular neighbourhood of $h(S^{\lambda-1})$ in $X$ in the sense of Definition \ref{def:halfneighbourhood}.
	
	An \emph{$n$--dimensional $\lambda$-handle}, is a copy of $D^k\times D^{n-k}$ which we attach to $\pd X$ along $\pd D^k\times D^{n-k}$ by an embedding
	\[
	\varphi : \pd D^k \times D^{n-k} \to \pd X.
	\]
	We call $\varphi$ an \emph{attaching map}, and canonically smooth corners \cite{GompStip} in the quotient space
	\[
	X\cup_\varphi h = \quotient{X \cup ( D^k\times D^{n-k} ) }{ \varphi(x) \sim x }
	\]
	to make $X\cup_\varphi h$ into a smooth manifold.
	The number $k$ is the \emph{index} of the handle $h$ and we call this process \emph{attaching a $k$-handle to $X$ with attachment map $\varphi$}.
	An $n$--dimensional manifold described by a handle decomposition whose handles have index no greater than $1$ is called an \emph{$n$-handlebody}.
\end{defn}




If $X\cup_\varphi h$ is a smooth manifold with attached $k$-handle, then we have special terms for the anatomy of $h$.
The $k$--disc at $D^k\times 0$ in $h$ is called the \emph{core} of $h$ and the $(n-k)$--disc $0 \times D^{n-k}$ is called the \emph{cocore}.
We call the $(k-1)$--sphere $\pd D^k \times 0$ the \emph{attaching sphere} and we call the $(n-k-1)$--sphere $ 0 \times \pd D^{n-k}$ the \emph{belt sphere}.
we call $\pd D^k\times D^{n-k}$, or its image through an attaching map $\varphi$, the \emph{attaching neighbourhood}.

\begin{rmk}
	If $\varphi$ and $\varphi'$ are isotopic attaching maps of the $k$--handle $h$ then the manifolds $X\cup_\varphi h$ and $X\cup_{\varphi'} h$ are diffeomorphic up to ambient isotopy.
	This follows from the Isotopy Extension Theorem.
	If $\varphi,$ $\varphi'$ are isotopic maps into the closed manifold $\pd X$, then the restrictions of these maps to the compact attaching neighbourhood are also isotopic hence induce an ambient isotopy $\Phi_t$.
	The diffeomorphism
	\[
		\begin{array}{cccc}
			\ident{[0,1]}\times\Phi: & [0,1]\times \pd X & \to & [0,1]\times\pd X \\
									 & (t,x) & \mapsto & (t,\Phi_t(x))
		\end{array}
	\]
	gives us the diffeomorphism $X\to X$ by identifying $[0,1]\times\pd X$ with a neighbourhood of $\pd X$ in $X$. \{CITE: RoSa\}
	
	We construct an attaching map $\varphi:\pd D^k\times D^{n-k}\to\pd X$ from an embedding of the attaching sphere $\varphi_0:\pd D^k\times 0\to \pd X$ and an identification $n$ of the normal bundle of the image of $\varphi$ with $\pd D^k \times \R^{n-k}$.
	From the Tubular Neighbourhood Theorem, this information determines an embedding of $\pd D^k\times D^{n-k}$ up to isotopy.
	Thus any attaching map is determined by
	\begin{enumerate}
	  \item an embedding $\varphi_0:S^{k-1}\to\pd X$ with trivial normal bundle, and
	  \item an identification $n$ of the normal bundle $\nu\varphi_0(S^{k-1})$ with $S^{k-1}\times\R^{n-k}$ which we call a \emph{framing of} $\varphi_0(S^{k-1})$
	\end{enumerate}
	up to isotopy, and the resulting manifold $X\cup_\varphi h$ up to diffeomorphism.
\end{rmk}

\begin{rmk}
\label{rmk:onehandle}
There are two important cases of handle attachment that we care about --- attachment of handles with index $1$ and $2$.
To attach a handle of index $1$ to an $(n+1)$--manifold $W$ over its $n$--manifold boundary $M$, we specify an embedding $\varphi_0$ of $S^0=\{-1,1\}$ into $M$.
We must also define a framing of $\varphi_0(S^{0})$ in $M$.
If $M$ is oriented, then a framing of $\varphi_0(\pm 1)$ is an element of $\gl{n}{\R}$.
Because an orientation of the $n$--manifold $M$ is defined by an association of an element of $\glp{n}{\R}$ to each point of $M$, we can say that a framing of $\varphi_0(\pm 1)$ is \emph{orientation preserving} if the chosen framing element is from $\glp{n}{\R}$ and \emph{orientation reversing} if the chosen framing element is from $\gln{n}{\R}$.
With these framings defined, we can form the attaching map $\varphi$ for our 1--handle $h$.
Because the attaching map $\varphi$ is defined on the disconnected space $S^0\times D^n$, we refer to $\varphi$ as a pair of attaching maps, one for each component. 
Note here that if $W$ is path connected and both framings are orientation reversing or orientation preserving, then the manifold $W\cup_\varphi h$ is nonorientable.
If exactly one framing is orientation preserving and the other orientation reversing then $W\cup_\varphi h$ is orientable.
\end{rmk}



Because our goal is to attach 2--handles to 4--manifolds, we focus now on embeddings of $\sone$ in 3--manifolds.
We can define a framing of an embedded $\sone$ in an oriented 3--manifold from a vector field over that embedded $\sone$ and the orientation of the 3--manifold.
Attaching a 2--handle to a 4--manifold consists of defining an embedding of $\sone\times D^2$ into the boundary of the 4--manifold.
This embedding is defined entirely by the image of the attaching sphere and by a framing of that image.
A framing is an association of every point of the attaching sphere with an element of $\gl{2}{\R}$.
We care about this embedding up to isotopy, so a framing can be reduced to an element of $\pi_1(\gl{2}{\R})=\pi_1(\sone)=\Z$.
A framing makes no sense without a zero framing as reference.
Once that is defined, any $n$--framing is also defined.
There is a canonical zero--framing of a knot in $\sthr$ found by the outward unit normal vector to any Seifert surface of that knot.
