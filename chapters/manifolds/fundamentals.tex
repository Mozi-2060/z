At its simplest, most colloquial definition, an $n$--dimensional manifold is a space that locally looks like real $n$--dimensional space or half space $\HN$.
We use \emph{charts} and \emph{atlases} to make explicit what is meant by ``looks like.''

\begin{defn}[Coordinates]
	\label{def:coordinates}
	Let $\HN\subset\RRN$ denote the closed real half space under the subspace topology, defined as
	\[
		\HN=\{(x_0,\dots,x_{n-1})\in\RRN : x_0\geq 0\}.
	\]
	We use the notations $\inter{\HN}$ and $\pd \HN$ to denote the interior and boundary of $\HN$ as subsets of $\RRN$ which, when $n>0$, are
	\[
		\begin{array}{ccccc}
			\inter{\HN} & = & \{(x_0,\dots,x_{n-1})\in\RRN : x_0 > 0\}, & &\\
			\pd \HN	& = & \{(x_0,\dots,x_{n-1})\in\RRN : x_0 = 0\} & \approx & \R^{n-1}.
		\end{array}
	\]
	When $n=0$, $\HH^0 = \R^0 = \{0\}$, so $\inter{\HH^0}=\R^0$ and $\pd\HH^0=\emptyset$.

	Let $X$ be a second-countable Hausdorff space.
	The pair $(U,f)$ where $U$ is an open subset of $X$ and $f$ is a homeomorphism from $U$ onto an open set of either $\RRN$ or $\HN$ is called a \emph{chart} of $X$.
	The map $f$ is a \emph{coordinate system} on $U$ and its inverse $f\inv$ is a \emph{parameterization} of $U$.
	Writing $f$ as
	\[
		f(u) = (\xi_0(u),\dots,\xi_{n-1}(u)),
	\]
	the functions $\xi_i$ are \emph{coordinate functions}
\end{defn}

\begin{defn}[Atlas]
	\label{def:atlas}
	Let $\mathcal{A}=\{(U_\alpha,f_\alpha)\}$ be a collection of charts of the space $X$ parameterized by $\alpha\in A$ for some indexing set $A$.
	If $\bigcup_A U_\alpha$ contains $X$, then $\mathcal{A}$ is an \emph{atlas} for $X$.
	The homeomorphisms $f_\alpha\comp f_\beta\inv:f_\beta(U_\alpha\cap U_\beta)\to f_\alpha(U_\alpha\cap U_\beta)$ are \emph{transition maps} of $\mathcal{A}$.
	We say that $(U_\alpha,f_\alpha)$ and $(U_\beta,f_\beta)$ are \emph{smoothly compatible} if either $U_\alpha\cap U_\beta$ is empty or the transition maps $f_\alpha\comp f_\beta\inv$ and $f_\beta\comp f_\alpha\inv$ are smooth as maps $\RRN\to\RRN$, i.e.\ they have continuous partial derivatives of all orders.
	If every pair of charts in an atlas is smoothly compatible then that atlas is a \emph{smooth atlas}.
	Two smooth atlases are equivalent if their union is a smooth atlas.
\end{defn}

\begin{defn}[Manifolds]
	\label{def:manifold}
	Let $X$ be a second-countable Hausdorff topological space, and $\mathcal{A}$ an atlas for $X$.
	If $\mathcal{A}$ is a smooth atlas, then the pair $(X,\mathcal{A})$ is a \emph{smooth n--manifold},  \emph{n--manifold}, or just \emph{manifold}.
	We usually omit writing the atlas when talking about a manifold.
	If $\mathcal{A}$ is a maximal smooth atlas then we call it a \emph{smooth structure} on $X$.
	We will assume that a smooth manifold is always equipped with a smooth structure.
\end{defn}

\begin{defn}[Boundary, Interior]	
	\label{def:boundary}
	Let $X$ be a smooth $n$--manifold.
	A chart $(U,f)$ is called an \emph{interior chart} is $f(U)$ is an open subset of $\RRN$ and is a \emph{boundary chart} if $f(U)$ is an open subset of $\HN$ with $f(U)\cap\pd \HN\neq\emptyset$.
	Let $p\in X$.
	We say that $p$ is an \emph{interior point} if it is in the domain of an interior chart, and is a \emph{boundary point} if it is in the domain of a boundary chart $(U,f)$ so that $f(p)\in\pd\HN$.
	The set of all boundary points of $X$ is called the \emph{boundary of $X$} and is denoted by $\pd X$.
	Similarly, the set of all interior points of $X$ is called the \emph{interior} of $X$ and is denoted by $\inter{X}$.
	If $X$ is compact with empty boundary then $X$ is \emph{closed} as a manifold.
	There is a potential conflict of this definition with the concept of topological closure.
	In the rare case that we want to say that a manifold is topologically closed, we will specify that the type of closure is topological.
	Otherwise, the statement of a ``closed manifold'' will refer to a manifold with empty boundary.
\end{defn}

\begin{ex}
	The $n$--sphere, denoted $S^n$ and defined as a subset of $\R^{n+1}$, is
	\[
	S^n = \{x\in\R^{n+1}:\norm{x}=1\},
	\]
	and the $(n+1)$--dimension closed disc, or $(n+1)$--disc, defined also as a subset of $\R^{n+1}$, is
	\[
	D^{n+1} = \{x\in\R^{n+1}:\norm{x}\leq 1\},
	\]
	Where $\norm{\,\cdot\,}$ is the euclidean norm defined for $x\in\R^{n+1}$ by
	\[
	\norm{x} = \Big( \sum x_i^2 \Big)^{1/2}.
	\]
	Note that the $(n+1)$--disc is an $(n+1)$--manifold with boundary, and that boundary is exactly the $n$--sphere, a closed $n$--manifold.
\end{ex}