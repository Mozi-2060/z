The main tool used to construct 4--manifolds in later chapters is handle attachment.
We focus mainly on the definitions and results needed to meaningfully attach 2--handles to a 4--manifold, using the methods that are explored in detail in \cite{Kosi93} to attach handles in way that ensures the resulting manifold is smooth.
This is accomplished by piping the areas to be joined together through cooked up diffeomorphisms.
We lay out the general theory behind joining manifolds along submanifolds of the boundary, then move into the specifics of handle attachment.

Let $\beta=(E,B,\pi)$ be a $k$-vector bundle with $B$ a compact $n$--manifold.
Let $\alpha:(0,\infty)\to(0,\infty)$ be an orientation reversing diffeomorphism.
Define a map $\alpha_E: E\setminus z(B) \to E\setminus z(B)$ by
\begin{equation}
\label{eqn:alpha}
\alpha_E(b,v) = \Big(b,\alpha(\norm{v})\frac{v}{\norm{v}}\Big).
\end{equation}
Let $M_1$, $M_2$ be $(n+k+1)$--manifolds with boundary, and let $f_1$, $f_2$ be embeddings of $E$ in $\pd M_1$ and $\pd M_2$.
Restricting the $f_i$ to $z(B)$, we get $f_i(z(B))$ as embedded submanifolds of the boundaries $\pd M_i$.
We may then view the $(E,f_i)$ as tubular neighbourhoods of $f(z(B))$ in $\pd M_i$, and can extend the neighbourhoods to $M_i$ as in Definition \ref{def:halfneighbourhood}, using two copies of $\HH^1$.

We can extend $\beta$ to a $(k+1)$--vector bundle $\mathcal{B}=(\mathcal{E},B,\Pi)$ and extend the $f_i$ to a pair of embeddings $F_1$, $F_2$ each defined on ``one-half'' of $\mathcal{E}$.
Then $F_i$ embeds its half of $\mathcal{E}$ in the boundary $\pd M_i$.
We obtain a new object by ``joining'' $M_1$ and $M_2$ along the embeddings $F_1$, $F_2$.
Explicitly, identify $(b,v)$ in $F_1(\mathcal{E}\setminus z(B))$ with $F_2\comp \alpha_\mathcal{E}\comp F_1\inv(b,v)$.
A space obtained in this manner will be denoted $M(f_1,f_2)$.

\begin{theorem}[Proposition \rom{6}.5.1 of \cite{Kosi93}]
	\label{thm:joinman}
	Let $\beta$, $M_i$, $f_i$ be as above, and let $M'=M(f_1',f_2')$ with $\restr{f_i}{B}=\restr{f_i'}{B}$, $i=1,2$.
	Then there is an automorphism $g$ of $E$ such that $M'$ is diffeomorphic to $M(f_1\comp g,f_2)$.
\end{theorem}

The machinery used to prove Theorem \ref{thm:joinman} is built up throughout Chapter \rom{6} of \cite{Kosi93}.
To concisely state the content of this theorem, we say that handle attachment over the chosen embeddings $f_1$, $f_2$ is \emph{unique up to bundle automorphisms of $E$}.
Now, we define handle attachment -- an explicit application of Theorem \ref{thm:joinman}.

Put $n=\lambda+\mu$ and let $M$ be an $n$--manifold with boundary.
Attaching an $n$--dimensional $\lambda$--handle to $M$ is the process of joining $M$ to $\DN$ along embeddings of a $\mu$--disc bundle over $S^{\lambda-1}$, into the boundaries of $M$ and $\DN$.
Write elements of $\RRN=\R^\lambda\times\R^\mu$ as $x=(x_\lambda,x_\mu)$, where $x_\lambda$ is the result of projecting $x$ onto the first $\lambda$ coordinates of $\RRN$ and $x_\mu$ is the result of projection onto the last $\mu$ coordinates.
This notation lets us write $\DN$, $\D^\lambda$, and $S^{\lambda-1}$ in $\RRN$ as
\[
\begin{array}{cccr}
\DN & = & \{x\in\RRN : \norm{x_\mu}^2 + \norm{x_\lambda}^2 \leq 1\}, & \\
\D^\lambda & = & \{x\in\RRN : x_\mu=\vec{0}, \norm{x_\lambda}^2 \leq 1\}, & \textrm{and} \\
S^{\lambda-1} & = & \{x\in \DN: x_\mu=\vec{0}, \norm{x_\lambda}=1\}. &
\end{array}
\]

\begin{defn}[Handle]
	% arctan isn't going to work, remember that x_\mu is a vector	
	Let $M$ be a smooth $n$--manifold with nonempty boundary and let $n=\lambda+\mu$.
	Let $S^{\lambda-1}\times \R^\mu$ be a trivial $\mu$--vector bundle.
	Define an embedding $f_2:S^{\lambda-1}\times\R^\mu\to S^{\lambda+\mu-1}=\pd\DN$ by
	\[
	f_2\big((x_\lambda,0),(0,x_\mu)\big) = \frac{(x_\lambda,\arctan x_\mu)}{(\norm{x_\lambda}^2+\norm{x_\mu}^2)^{1/2}},
	\]
	and let $f_1:S^{\lambda-1}\times\D^\mu\to \pd M$ be an embedding such that 
	
	Let $f:S^{\lambda-1}\to \pd M$ be an embedding with proper tubular neighbourhoods $\nu_{\pd M}\,f$ and $\nu_M\,f$ and $F:T\to M$ an extension of $f$ such that $\restr{F}{S^\lambda}$ is a vector bundle isomorphism to a proper tubular neighbourhood of $f(S^{\lambda-1})$ in $\pd M$, and $F(T)$ is a proper tubular neighbourhood in the sense of Definition \ref{def:halfneighbourhood} .
	We obtain a new space by identifying $x\in T\setminus h(S^{\lambda-1})$ with $F\comp\alpha(x)\in M\setminus f(S^{\lambda-1})$, a process we call \emph{attaching an $n$--dimensional $\lambda$--handle to $M$ along $f(S^{\lambda-1})$}.
	We denote the new space by $M\cup_f H^\lambda$.	
\end{defn}

Attaching a handle is a special case of joining a pair of manifolds along submanifolds of the boundary.
Theorem \ref{thm:joinman} applies, so attachment of a handle over $f$ is unique up to a bundle automorphism.
Our uniqueness theorem for proper tubular neighbourhoods applies as well, so attachment of a handle is unique up to ambient isotopy of the embedding $f$.
In other words, the characteristics of handle attachment that fully describe the smooth $n$--manifold $M\cup_f H^\lambda$ up to diffeomorphism are:
\begin{enumerate}
	\item The isotopy class of the embedding $f:S^{\lambda-1}\to\pd M$ where the proper tubular neighbourhood $\nu_{\pd M}\,f(S^{\lambda-1})$ has trivial structure as the normal bundle $N_{\pd M}\,f(S^{\lambda-1})$, and 
	\item the isotopy class of the extension of $f$ to $F: S^{\lambda-1}\times R^\mu\to \nu_{\pd M}\,f(S^{\lambda-1})$, an explicit trivialization of the normal bundle.
\end{enumerate}

There is some specific language that is useful to the description of handle attachment.

\begin{defn}
	Let $M\cup_h H^\lambda$ be an $n$--manifold with a $\lambda$--handle attached.
	The embedding $f:S^{\lambda-1}\to\pd M$ is called the \emph{attaching map}, its image $f(S^{\lambda-1})$ is the \emph{attaching sphere}, and its tubular neighbourhood is the \emph{attaching neighbourhood}.
	The explicit trivialization $F: S^{\lambda-1}\times \R^\mu\to \nu_{\pd M}\,h(S^{\lambda-1})$ is called a \emph{normal framing} or just \emph{framing}.
	Identify $M\setminus f(S^{\lambda})$ and $\DN\setminus S^{\lambda-1}$ with their images in $M\cup_f H^\lambda$.
	Then $\DN\setminus S^{\lambda-1}$ is the \emph{handle}, the disc $\D^\lambda=\DN\cap(\R^\lambda\times\{\vec{0}\})$ is the \emph{core} of the handle, and the disc $\D^\mu=\DN\cap(\{\vec{0}\}\times\R^\mu)$ is the \emph{cocore} of the handle.
	The boundary $S^{\mu-1}$ of the cocore is the \emph{belt sphere}.
\end{defn}

There are two special cases of handle attachment to discuss.
First, let $M$ be a smooth, orientable, path--connected $n$--manifold.
The attaching sphere of a 1--handle is $S^0=\pd \D^1$, which is a pair of points.
For $\pd M$ compact, connected, and nonempty, there is a unique isotopy class of embeddings $f:S^0\to\pd M$.
This means that $M\cup_f H^1$ is determined entirely by the extension of $f$.
The normal bundle is a $(n-1)$--vector bundle over $S^0$, so it is vacuously trivial.
A trivialization of $\R^{n-1}\times S^0$ can be written as a pair of elements $A_{-1}$, $A_1$ from $\gl{n-1}{\R}$, each acting on one of the connected components of $\R^{n-1}\times S^0$.
The group $\gl{n-1}{\R}$ has two path--connected components, denoted $\gln{n-1}{\R}$ and $\glp{n-1}{\R}$, which are the spaces of negative and positive determinant transformations respectively.
Every element of $\gl{n-1}{\R}$ is an embedding $\R^{n-1}\to\R^{n-1}$, so any path in $\gl{n-1}{\R}$ is an isotopy of its endpoints.
It is then easy to see that are four isotopy classes of our trivialization, and those fall into two types.
Either both $A_{-1}$ and $A_1$ are are orientation preserving (reversing), or $A_{-1}$ is orientation preserving (reversing) and $A_1$ is orientation reversing (preserving).
Under the first type of automorphism, $M\cup_f H^1$ is a non-orientable manifold.
Under the second, $M\cup_f H^1$ is orientable.

Next, let $W$ be a 4--manifold with orientable boundary $M$.
To attach a 2--handle to $W$, we need an embedding $f:S^1\to M$ and a framing of the normal bundle $N_M f(S^1)$.
The normal bundle is homeomorphic to $S^1\times D^2$, so it is what we call a solid torus.

\begin{defn}
	A space $V$ that is homeomorphic to $S^1\times\DD$ is called a \emph{solid torus}.	
	A simple closed curve $J$ in $\pd V$ that bounds a 2--disc in $V$ is called a \emph{meridian}.
	A simple closed curve $K$ in $\pd V$ that intersects a meridian at a single point is called a \emph{longitude}.
	Note that there are infinitely many isotopy classes of longitudes of $V$, whereas there's exactly one isotopy class of meridians.
\end{defn}

Once an embedding $f:S^1\to M$ is chosen, $W\cup_f H^2$ is determined by the framing $F:S^1\times \R^2\to \nu_{\pd M}\,f(S^1)$.
To investigate possible framings, we consider the \emph{mapping class group} of the solid torus.
The mapping class group of a space $X$ is the group of isotopy classes of automorphisms $X$, and it is denoted $mpg(X)$.
For $V$ a solid torus, $mpg(V)$ is very closely linked to the mapping class group of the torus $T^2 = S^1\times S^1=\pd V$, which is the special linear group $\textrm{SL}_2(\Z)$, the group of $2\times 2$ matrices with integer entries and determinant $\pm 1$.
\begin{lem}
	Let $V$ be an open solid torus and let $f$ be an automorphism of the torus $\pd V$.
	Then $h$ extends to an automorphism of $V$ if and only if $f$ maps a meridian to a meridian.	
\end{lem}

\begin{lem}
	A pair of automorphisms $f,g$ of $V$ that agree on $\pd V$ and map a meridian to a meridian are isotopic.	
\end{lem}

\begin{theorem}
	The mapping class group of a solid torus $V$ is the subgroup of the mapping class group of $\pd V$ containing automorphisms that map meridians to meridians.
	This subgroup is isomorphic to $\Z$.	
\end{theorem}

\begin{proof}
	In \cite{Rolf76}, it is shown that the mapping class group of $T^2$ is $\textrm{SL}_2(\Z)$ by generating the group from a pair of basic automorphisms of $T^2$ called \emph{twists}, and the swapping map $(z,w)\to(w,z)$.
	The twists are defined as:
	\[
	\begin{array}{ccccc}
	h_L(e^{i\theta},e^{i\phi}) & = & (e^{i(\theta+\phi)},e^{i\phi}) & & \textrm{``longitudinal twist''} \\
	
	h_M(e^{i\theta},e^{i\phi}) & = & (e^{i\theta},e^{i(\theta+\phi)}) & & \textrm{``meridinal twist''}	
	\end{array}
	\]
	Fix the meridinal and longitudinal directions of the boundary of an open solid torus $V$ to coincide with the meridinal and longitudinal directions of $V$.
	This would mean that we write $V=S^1\times D^2$, with $\{1\}\times S^1$ a meridian and $S^1\times\{1\}$ a longitude.
	Notice that the meridinal twist maps a meridian to a meridian, so it extends to an automorphism of $V$.
	Also, neither the longitudinal twist nor the swap map a meridian to a meridian, so neither extend to an automorphism of $V$.
	
	Let $F$ be an automorphism of $V$ with restriction $f$ to $\pd V$.
	Because $f$ is an automorphism of $\pd V$, it can be written as the product of twists and swaps.
	Because $f$ preserves meridians, it can be written entirely as a power of the meridinal twist.
	In the other direction, any automorphism that is the power of a meridinal twist clearly preserves meridians.
	Because these automorphisms are written as powers of $h_M$, there is a clear isomorphism from this subgroup to $\Z$.
\end{proof}

The content of these results means that the attachment of a 2--handle to a 4--manifold can be described by an embedding $f:S^1\to M$ and an integer $k$ called the \emph{framing constant} that determines the framing automorphism as long as we know what a framing constant of $0$ means.
To see this, first notice that $h_M$ can be extended to
\[
\begin{array}{cccc}
H_M: & S^1\times\DD & \to & S^1\times\DD \\
& (e^{i\theta}, re^{i\phi}) & \mapsto & (e^{i\theta},re^{i(\theta+\phi)})	
\end{array}
\]
which represents a generator of $mpg(S^1\times\DD)$.


restricts to , and can be pre-- and post--composed by the homeomorphism
\[
\begin{array}{cccl}
\varphi: & S^1\times D^2 & \to & S^1\times \R^2 \\
& (e^{i\theta}, re^{i\phi}) & \mapsto & (e^{i\theta}, -\ln(1-r)e^{i\phi})
\end{array}	
\]
and its inverse to obtain a homeomorphism $S^1\times \R^2\to S^1\times \R^2$ that represents a generator of $mpg(S^1\times\R^2)$.
Because $\nu_{\pd M}\,f(S^1)$ is homeomorphic to $S^1\times\R^2$, a one-to-one correspondence
$\psi$ between $mpg(S^1\times\R^2)$ and the isotopy classes of maps $S^1\times\R^2\to\nu_{\pd M}\,f(S^1)$ is determined entirely by $\psi(\id)$.

When $M$ is $S^3$, $f:S^1\to S^3$ is a \emph{knot}, and $\nu_M\,f(S^1)$ is an open solid torus.
There is choice of longitude for 




Put $n=\lambda+\mu$ and let $M$ be an $n$--manifold with boundary.
Attaching an $n$--dimensional $\lambda$--handle to $M$ is the process of joining $M$ to $\DN$ along embeddings of $\beta=(E,S^{\lambda-1},\pi)$, a $\mu$--vector bundle over $S^{\lambda-1}$, into the boundaries of $M$ and $\DN$.
%An explicit embedding of $E$ into $\pd D^n$ is explored in depth in \rom{6}.6 of \cite{Kosi93}, so we will touch only on the facts necessary to explicitly define handles.

Write elements of $\RRN=\R^\lambda\times\R^\mu$ as $x=(x_\lambda,x_\mu)$, where $x_\lambda$ is the result of projecting $x$ onto the first $\lambda$ coordinates of $\RRN$ and $x_\mu$ is the result of projection onto the last $\mu$ coordinates.
This notation lets us write $S^{\mu-1}$ in $\RRN$ as
\[
	S^{\mu-1} = \{x\in \DN: x_\lambda=\vec{0},\norm{x_\mu}=1\}.
\]
For $\varepsilon\in [\,0,1)$ define $T(\varepsilon)$ as a subset of $\DN$ by
\[
	T(\varepsilon) = \{x\in \DN: \norm{x_\lambda}^2>\varepsilon\}.
\]
Note that $T(\varepsilon)$ collapses onto $S^{\lambda-1}$ via the projection $(x_\lambda,x_\mu)\mapsto x_\lambda/\norm{x_\lambda}$.
Abbreviate $T(0)$ as $T$.
Define a map
\[
	\alpha:T(\varepsilon)\setminus S^{\lambda-1} \to T(\varepsilon)\setminus S^{\lambda-1}
\]
defined by
\begin{equation}
	\alpha(x_\lambda, x_\mu) =
	\Bigg( 
		\frac{x_\lambda}{\norm{x_\lambda}}
		(1-\norm{x_\lambda}^2+\varepsilon)^{1/2},
		x_\mu
		\frac{
			(\norm{x_\lambda}^2-\varepsilon)^{1/2}
		}{
			(1-\norm{x_\lambda}^2)^{1/2}
		}
	\Bigg).
\end{equation}
To relate handle attachment with our initial definition of joining manifolds over submanifolds in the boundary, first notice that $T(\varepsilon)$ has the structure of ``one-half'' of a $(\mu+1)$--vector bundle over $S^{\lambda-1}$.
Next, $\alpha$ is a composition of the diffeomorphism $\DN\setminus S^{\lambda-1}\to D^\lambda\times \D^\mu$ given by
\begin{equation}
	\label{eqn:diffeo}
	(x_\lambda,x_\mu)\mapsto \Big(x_\lambda,\frac{x_\mu}{(1-\norm{x_\lambda}^2)^{1/2}}\Big)
\end{equation}
with an involution on $(D^\lambda\setminus\{\vec{0}\})\times \D^\mu$ given by
\begin{equation}
	(x_\lambda,x_\mu)\mapsto \Big(\frac{x_\lambda}{\norm{x_\lambda}}(1-\norm{x_\lambda}^2+\varepsilon)^{1/2},x_\mu \Big)
\end{equation}
and then with the inverse of equation \ref{eqn:diffeo}.
Thus, $\alpha$ is an orientation reversing diffeomorphism as in equation \ref{eqn:alpha}.