A common tool in both the construction and description of a manifold is a \emph{bundle}.
The ultimate construction in this section is the tubular neighbourhood of a submanifold, which is unique up to a fibre-preserving isotopy.

\begin{defn}[Bundles]
	A bundle is most concisely represented by the composition
	\[
		F\into E\overset{\pi}{\onto} B,
	\]
	where $B$ is called the \emph{base space}, $E$ the \emph{total space}, $F$ the \emph{fibre}, and $p$ the \emph{projection}.
	There are some 
	For the tuple $\beta = (F,E,B,\pi)$ to be a fibre bundle, we require that $\pi$ be continuous, that the subspaces $F_x = \pi\inv(x)$ are each homeomorphic to $F$, and that for every point $x\in B$ there exists an open neighbourhood $U\subset B$ containing $x$ and a homeomorphism
	\[
		\varphi: U\times F\to \pi\inv(U)
	\]
	so that $\pi\comp\varphi\,(y,v)=y$ for any pair $(y,v)$ in $U\times F$.
	Such a homeomorphism is called a \emph{local trivialization}.
	Any point $p\in E$ may be represented uniquely by a pair $p=(x,v)$ with $x\in B$, $v\in F$, and $\pi(p)=x$.
	When $F$ is a vector space, we call $\beta$ a \emph{vector bundle}.
	A function $s:B\into E$ so that $s(x)\in F_x$ for every $x\in B$ is called a \emph{section}.
	When $\beta$ is a vector bundle, we can define the \emph{zero section} of $\beta$ to be the map
	\[
		\begin{array}{cccc}
			z: & B & \to 	 & E \\
			   & x & \mapsto & (x,0).
		\end{array}
	\]
	We also call the subspace $z(B)\subset E$ the zero section of $E$.
\end{defn}

\begin{defn}[Bundle Isomorphism]
	Let $\Phi:\beta_0\to \beta_1$ be a map between vector bundles.
	We call $\Phi$ a \emph{fibre map} if $\Phi:E_0\to E_1$ covers a map $\varphi:B_0\to B_1$.
	For $\Phi$ to cover $\varphi$, that means the following diagram commutes:
	\[
		\begin{tikzcd}
			E_0 \arrow{r}{\Phi} \arrow[swap]{d}{\pi_0} & E_1 \arrow{d}{\pi_1} \\
			B_0 \arrow{r}{\varphi} & B_1
		\end{tikzcd}
	\]
	This means that if $x\in B_0$ and $\varphi(x)=y$, then $\Phi$ maps $F_x$, the fibre over $x$, to $F_y$, the fibre over $y$ by a map we will denote $\Phi_x$.
	If $\Phi_x$ is a linear map for every $x\in B_0$, then we call $\Phi$ a \emph{bundle morphism}.
	If $\Phi$ is a bundle morphism, $B_0=B_1=B$, $\Phi_x$ bijective for each $x\in B$, and $\varphi$ is $\ident{B}$, then $\Phi$ is a \emph{bundle isomorphism}.
\end{defn}

\begin{defn}[Tangent Bundle]
	Let $X$ be an $n$--manifold.
	We define the total space $TX$ of a vector bundle with base space $X$ and fibres $\RRN\approx T_p X$ at any $p\in X$ by taking the disjoint union of all tangent spaces:
	\[
		TX = \bigsqcup_{p\in X} T_p X.
	\]
	The projection $\pi:TX\to X$ is defined by $\pi(q)=p$ for every $q\in T_p X$.
	We call $TX$ the \emph{tangent bundle} over $X$.
\end{defn}

\begin{defn}[Normal Bundle]
	Let $X$ be a $k$--dimensional submanifold of the $n$--manifold $Y$.
	At a point $p\in X\subset Y$, the tangent space $T_p X$ is a subspace of the tangent space $T_p Y$.
	Denote the orthogonal complement to $T_p X$ in $T_p Y$ by $N_p X$ and call it the normal space at $p$ in $X$.
	That is, $T_p X\oplus N_p X = T_p Y$.
	We define the total space $NX$ of a vector bundle with base space $X$ and fibres $\RRN\approx T_p X$ at any $p\in X$ by taking the disjoint union of all normal spaces:
	\[
		NX = \bigsqcup_{p\in X} N_p X.
	\]
	The projection $\pi:NX\to X$ is defined by $\pi(q)=p$ for every $q\in N_p X$.
	We call $NX$ the \emph{normal bundle} over $X$.	
\end{defn}

\begin{defn}[Tubular Neighbourhood]
	\label{def:tubularneighbourhood}
	Let $X\subset Y$ be a submanifold with $\pd X = \pd Y = \emptyset$.
	An embedding $f:NX\to Y$ with $f(x,0)=x$ and $f(NX)$ an open neighbourhood of $X$ in $Y$ is called a \emph{tubular neighbourhood} of $X$ in $Y$.
	We often denote the pair $(X,f)$ by $\nu X$.
\end{defn}

\begin{theorem}[Tubular Neighbourhood, Theorem 4.5.2 in \cite{Hirsch67}]
	\label{thm:tubularneighbourhood}
	Let $X$ be a smooth closed submanifold of the smooth closed manifold $Y$.
	Then $X$ has a tubular neighbourhood in $Y$.
\end{theorem}

Similar to the tubular neighbourhood is the collar of a manifold with boundary.

\begin{prop}[Collars]
	\label{defthm:collar}
	Let $X$ be a manifold with nonempty boundary $\pd X$.
	There exists an open neighbourhood $U$ of $\pd X$ in $X$ and a diffeomorphism
	\[
	f: U \to \pd X\times[\,0,1)
	\]
	where $f(\pd X)=\pd X\times \{0\}$.
	The neighbourhood $U$ is called a \emph{collar neighbourhood} or \emph{collar} of $\pd X$.	
\end{prop}

Tubular neighbourhoods are unique up to a fibre-preserving isotopy, so lets define precisely what that means.

\begin{defn}[Homotopy]
	\label{def:homotopy}
	Let $f,g:X\to Y$ be smooth maps between smooth manifolds.
	Denote the closed unit interval $[0,1]$ by $\I$.
	A function 
	\[
		\begin{array}{cccc}
			H: & X\times \I & \to & Y \\
			   & (x,t)	& \mapsto & H_t(x)
		\end{array}
	\]
	with $H_0(x)=f(x)$ and $H_1(x)=g(x)$ is a \emph{homotopy} between $f$ and $g$.
	If a homotopy exists, then $f$ and $g$ are \emph{homotopic}.
	Less formally, $f$ and $g$ being homotopic means that one can be continuously deformed into the other.
	The topological spaces $X$ and $Y$ are \emph{homotopy equivalent} if there exist continuous maps $f:X\to Y$ and $g:Y\to X$ for which $g\comp f$ is homotopic to $\ident{X}$ and $f\comp g$ is homotopic to $\ident{Y}$.
	
	Because we are primarily interested in smooth functions to build our machinery, we extend our definition of homotopy to a smooth version.
	With the notation above, a smooth map $H:X\times[0,1]\to Y$ with $H_0(x)=f(x)$ and $H_1(x)=g(x)$ is a \emph{smooth homotopy} between $f$ and $g$.
	If a smooth homotopy exists, then $f$ and $g$ are \emph{smoothly homotopic}.
\end{defn}

A homotopy through embeddings is called an isotopy.
This is useful for comparing embedded submanifolds.

\begin{defn}[Isotopy]
	\label{def:isotopy}
	Let $X$ be a smoothly embedded submanifold of $Y$.
	An \emph{isotopy} of $X$ in $Y$ is a smooth homotopy
	\[
		\begin{array}{crcl}
			F: & X\times \I & \to & Y \\
			   & F(x,t) & = & F_t(x)
		\end{array}
	\]
	such that the related map
	\[
		\begin{array}{cccc}
			\hat{F}: & X\times\I & \to & Y\times\I \\
					 & (x,t) & \mapsto & (F_t(x),t)
		\end{array}
	\]
	is an embedding.
	The submanifolds $F_0(X)$ and $F_1(X)$ are \emph{isotopic}.
	When $X=Y$ and $F_t$ is a diffeomorphism for each $t$, $F$ is a \emph{diffeotopy} of $Y$.
	
	Let $X$ be a smoothly embedded submanifold of $Y$ and consider a pair of tubular neighbourhoods $f,g:NX\to Y$ of $X$ in $Y$.
	An isotopy
	\[
		\begin{array}{crcl}
			F: & NX\times\I & \to & Y \\
			   & F(x,t) & = & F_t(x)
		\end{array}
	\]
	satisfying the following properties:
	\begin{enumerate}
		\item $F_0=f$ and $F_1=g$,
		\item $F_0(NX)=F_1(NX)$,
		\item $F_1\inv\comp F_0$ is a vector bundle isomorphism $NX\to NX$,
	\end{enumerate}
	 is an \emph{isotopy of tubular neighbourhoods}, and the tubular neighbourhoods $f$ and $g$ are \emph{isotopic}.
\end{defn}

This leads into our uniqueness result for tubular neighbourhoods.

\begin{prop}[Uniqueness of Tubular Neighbourhoods, Theorem 4.5.3 in \cite{Hirsch67}]
	\label{prop:uniquenesstubularneighbourhood}
	Let $X\subset Y$ be a submanifold with $\pd X=\pd Y=\emptyset$.
	Any pair of tubular neighbourhoods of $X$ in $Y$ are isotopic.
\end{prop}	

\begin{theorem}[Isotopy Extension]
	\label{thm:isotopyextension}
	Let $X$ be a smooth compact submanifold of the smooth closed manifold $Y$, and let
	\[
		\begin{array}{crcl}
			F: & X\times\I & \to & Y \\
			   & F(x,t) & = & F_t(x)
		\end{array}
	\]
	be an isotopy of $X$ in $Y$.
	Let $L$ be the subset of $Y$ equal to the union of the images of $F_t$ for each $t$.
	More precisely,
	\[
		L = \bigcup_{t\in[0,1]} F_t(X).
	\]
	There exists a diffeotopy 
	\[
		\begin{array}{crcl}
			G: & Y\times\I & \to & Y \\
			   & G(y,t) & = & G_t(y)				
		\end{array}
	\]
	with $G_0=\ident{Y}$, $G_1$ equal to $F_1$ on $X\subset Y$, and $F_t$ the identity on $Y$ outside of an arbitrarily small neighbourhood of $L$ for all $t$.
\end{theorem}

\begin{defn}[Ambient Isotopy]
	\label{def:ambientisotopy}
	Let $X$, $Y$, $F$ be as above.
	The isotopy $G:Y\times\I\to Y$ guaranteed by Theorem \ref{thm:isotopyextension} is called an \emph{ambient isotopy}.
	The images $F_0(X)$ and $F_1(X)$ are \emph{ambiently isotopic} as submanifolds of $Y$.
\end{defn}

Our tubular neighbourhoods above were defined only for closed submanifolds of closed manifolds.
We can extend our definitions and results to include manifolds with boundary as long as the boundaries satisfy some niceness conditions.

\begin{defn}[Neat Submanifold]
	We use here $\ttilde{\HH}^m$ to denote the subspace of $\HN$ for which the last $n-m$ coordinates are $0$. 
	Let $X\subset Y$ be a $m$--submanifold of the $n$--manifold $Y$ satisfying the following conditions:
	\begin{enumerate}
		\item $X$ is a topologically closed subset of $Y$,
		\item $X\cap \pd Y=\pd X$, and
		\item for every $x\in\pd X$, there is a chart $(U,f)$ of $Y$ with $f:U\to\HN$ such that $f\inv(\tilde{\HH}^m)=U\cap X$.
	\end{enumerate}
	Then $X$ is a \emph{neat} submanifold of $Y$.
	The last condition for neatness essentially guarantees that $\pd X$ meets $\pd Y$ the same way that $\tilde{\HH}^m$ meets $\HN$.
\end{defn}

\begin{defn}[Neat Tubular Neighbourhood]
	Let $X$ be a neat submanifold of $Y$.
	An embedding $f:N(\inter{X})\to\inter{X}$
	
		Let $X\subset Y$ be a submanifold with $\pd X = \pd Y = \emptyset$.
		An embedding $f:NX\to Y$ with $f(x,0)=x$ and $f(NX)$ an open neighbourhood of $X$ in $Y$ is called a \emph{tubular neighbourhood} of $X$ in $Y$.
		We often denote the pair $(X,f)$ by $\nu X$.
	
	
\end{defn}