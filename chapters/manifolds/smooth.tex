\begin{defn}[Smooth Map]
	\label{def:smoothmap}
	Let $(X,\{U_\alpha,f_\alpha\})$ and $(M,\{V_\beta,g_\beta\})$ be smooth $n$-- and $k$--manifolds respectively and let $\varphi:X\to M$ be a map between them.
	If, for any $\alpha$ and $\beta$, the composition $g_\beta\comp \varphi\comp f_\alpha\inv$ is smooth as a map $\RRN\to\R^k$, then we say $\varphi$ is \emph{smooth} as a map between manifolds.
	If $\varphi$ is smooth and a well-defined $\varphi\inv$ exists and is smooth, then $\varphi$ is called a \emph{diffeomorphism} between manifolds.
	We say that manifolds are \emph{diffeomorphic} if there exists a diffeomorphism between them.
\end{defn}

\begin{prop}
	\label{prop:diffeoequiv}
	Diffeomorphism is an equivalence relation on the space of smooth manifolds.
\end{prop}

Manifolds are defined by their local homogeneity, and a tangent space is a precise description of that homogeneity near a point.

\begin{defn}[Tangent Space]
	\label{def:tangentspace}
	Let $M$ be an $n$--manifold and $p$ be a point in $M$.
	A smooth map $\gamma:(-\varepsilon,\varepsilon)\to M$ with $\gamma(0)=p$ is called a \emph{curve} in $M$ through $p$.
	Note that this definition is independent of the chart used, as all of our transition maps are smooth.
	
	For $p$ in the boundary of $M$, we allow an additional type of curve in our definition.
	Let $\gamma$ be a smooth map $(-\varepsilon,0\,] \to M$ or $[\,0,\varepsilon) \to M$ each with $\gamma(0)=p$.
	A map of the first type is called an \emph{outward curve} in $M$ through $p$ and a map of the second type is called an \emph{inward curve} in $M$ through $p$.
	
	Let $C_p M$ be the space of smooth curves in $M$ through $p$ and $\gamma_1, \gamma_2$ elements of $C_p M$.
	We can define an equivalence relation $\sim$ on $C_p(M)$ by saying that $\gamma_1\sim\gamma_2$ if
	\[
		\Ddt(f\comp\gamma_1)(0) = \Ddt(f\comp\gamma_2)(0).
	\]
	An equivalence class of the curve $\gamma$ in $C_p M$ is a \emph{tangent vector} at $p$ and is written as $\gamma'(0)$.
	The quotient $C_p M/\sim$ is the \emph{tangent space} at $p$, denoted $T_p M$.
	A vector $\gamma'(0)$ of $T_p M$ is called an \emph{inward vector} if the equivalence class $[\gamma]$ contains only inward curves, and an \emph{outward vector} if $[\gamma]$ contains only outward curves.
\end{defn}

\begin{prop}
	\label{prop:tangentspacevectorspace}
	Let $M$ be an $n$--manifold and $p$ a point in $M$.
	Then $T_p M$ is a vector space isomorphic to $\RRN$.
\end{prop}

\begin{proof}
	Let $(U,f)$ be a chart containing $p$ with $f(p)=\vec{0}$.
	It follows from our definition of tangent space that the map defined by
	\[
		\begin{array}{crcl}
			df : & T_p M & \to & \RRN \\
			& \gamma'(0) & \mapsto & \Ddt(f\comp \gamma)(0) .
		\end{array}
	\]
	is a bijection, so we also have a well defined inverse $(df)\inv$.
	We define operations on $T_p M$ so that $F$ is strengthened to a vector space isomorphism:
	\[
		\begin{array}{rcl}
			\gamma_1'(0)+\gamma_2'(0) & = & (df)\inv(df(\gamma_1'(0))+df(\gamma_2'(0))), \\
			t\gamma_1'(0) & = & (df)\inv(t\:df(\gamma_1'(0))).
		\end{array}
	\]
	The vector space structure of $T_p M$ is independent of the choice of chart.
	To see this, let $L_v$ be the parameterized straight line through $\vec{0}$ whose velocity is $v\in\RRN$.
	Precisely, $L_v(t)=tv.$
	For any chart $(U,f)$ with $f(p)=0$, the curve $f\inv\comp L_v$ is a curve through $p$ for which $df(f\inv\comp L_v)=v$.
	Because an equivalence class contains such a curve for any applicable chart, the association of a tangent vector with a vector in $\RRN$ does not depend on the chart used.
\end{proof}

In normal calculus, the derivative is a linearization of a function.
We use tangent spaces to define the derivative of a smooth map between manifolds because tangent spaces are used as pointwise linearizations of manifolds.

\begin{defn}[Differential]
	Let $\varphi:X\to M$ be a smooth map between manifolds.
	Let $p\in X$ and $\varphi(p)=q\in M$.
	The \emph{differential} of $\varphi$ at $p$ is a linear map defined as
	\[
		\begin{array}{crcl}
			d\varphi_p: & T_p X & \to & T_q M\\
					  & \gamma'(0) & \mapsto & (\varphi\comp\gamma)'(0),
		\end{array}
	\]
	and the element $(\varphi\comp\gamma)'(0)$ of $T_q M$ is the \emph{pushforward} of the tangent vector $\gamma'(0)$.
\end{defn}

We also classify smooth maps between manifolds.
This classification helps us define a submanifold and provides some of the groundwork for defining handles in the next chapter.

\begin{defn}[Embedding]
	Let $\varphi:X\to M$ be a smooth map between manifolds.
	The \emph{rank} of $\varphi$ at the point $p\in X$ is the rank of the differential $d\varphi$.
	This is computed as the rank of the Jacobian matrix of $\varphi$ under a coordinate system or as the dimension of $\varphi(T_p X)\subset T_q M$.
	If $\varphi$ has rank $k$ for every $p$ in $X$, then $\varphi$ is of \emph{constant rank} and we say $\rk\varphi=k$.
	 
	If $d\varphi$ is injective at $p$, then $\varphi$ is \emph{immersive} at $p$.
	If $\varphi$ is an everywhere immersive, then it is an \emph{immersion}.
	This is equivalent to $\rk{\varphi}=\dimn X$.
	If $d\varphi$ is surjective at $p$, then $\varphi$ is \emph{submersive} at $p$, and $p$ is a \emph{regular point} of $\varphi$.
	If $\varphi$ is everywhere submersive, then it is a \emph{submersion}.
	This is equivalent to $\rk{\varphi}=\dimn M$.
	If $\rk d\varphi_p$ is less than maximal, then $p$ is a \emph{critical point} of $\varphi$.
	In this case, $\varphi(p)=q$ is a critical value.
	If $p$ is a critical point and $d^2\varphi_p$ is of less than full rank, then $p$ is a \emph{degenerate critical point}.
	A smooth map with no degenerate critical points is called \emph{generic}.
	A point $q\in M$ whose preimage $\varphi\inv(q)$ consists entirely of regular points is called a \emph{regular value}.
	
	If $\varphi$ is an injective immersion that is a homeomorphism onto its image $\varphi(X)\subset M$ under the subspace topology, then it is a \emph{smooth embedding} or just \emph{embedding}.
	
	Let $X\subset M$ where $M$ is a manifold, and let $i:X\into M$ be inclusion.
	If $i$ is an embedding, then we call $X$ an \emph{embedded submanifold} or \emph{submanifold} of $M$.
	One can consider $X$ to be a manifold with smooth structure induced by the embedding.
\end{defn}

To clarify what it means for a smooth function to be generic, consider a smooth function with degenerate critical points.
Through small perturbations, these degenerate critical points can be turned into non-degenerate critical points.
Because such a function could be easily 'fixed,' we say that a generic smooth map is one whose critical points are all non-degenerate.

\begin{theorem}[Regular Value Theorem]
	Let $\varphi:X\to M$ be a smooth map between manifolds.
	Let $q\in M$ be a regular value.
	The preimage $\varphi\inv(q)$ is either empty or a $(\dimn{X}-\dimn{M})$--submanifold of $X$.
\end{theorem}

\begin{prop}
	\label{prop:boundariesaremanifolds}
	Let $M$ be an $(n+1)$--manifold.
	The boundary $\pd M$ is an $n$--dimensional closed submanifold of $M$.
\end{prop}

The main result of this work is applicable to orientable 3--manifolds, so we will review what is meant by a space being orientable.
Essentially, orientability is a guarantee that when you go for a walk your right and left sides haven't switched places by the time you get home.

\begin{defn}
	\label{def:orientation}
	Let $V^n$ be an $n$--dimensional real vector space, $\mathcal{B}(V^n)$ the set of ordered bases for $V^n$, and $\gl{n}{\R}$ the general linear group, i.e.\ the space of $n\times n$ invertible matrices with entries in $\R$.
	Let $b_1$ and $b_2$ be any two ordered bases from $\mathcal{B}(V^n)$.
	It is a standard result of linear algebra that there is a unique element $A$ of $\gl{n}{\R}$ that transforms $b_1$ into $b_2$.
	If the determinant $\det A$ is positive, then $b_1$ and $b_2$ are \emph{positively oriented} with respect to each other.
	If $\det A$ is negative, then $b_1$ and $b_2$ are \emph{negatively oriented}.
	We can define an equivalence relation $\sim$ on $\mathcal{B}(V^n)$ by saying that $b_1\sim b_2$ if they are positively oriented.
	An \emph{orientation} of $V^n$ is a choice of one of the two equivalence classes of $\mathcal{B}(V^n)$.
	This also allows us to classify linear transformations of $\gl{n}{\R}$ by their action on the quotient space $\mathcal{B}(V^n)/\sim$ by saying they are \emph{orientation preserving} if they have positive determinant and \emph{orientation reversing} if they have negative determinant.
\end{defn}

Orientations of $\RRN$ let us define orientations of manifolds.

\begin{defn}
	Suppose we have fixed an orientation of $\RRN$.
	Let $M$ be an $n$--manifold.
	We say that an \emph{orientation} of $M$ is a consistent choice of orientation of the tangent space at every point of $M$.
	A consistent choice of orientation means that for every chart $(U,f)$ with $f:U\to\RRN$, the vector space isomorphism $df:T_p M\to \RRN$ is orientation preserving at every point in $U$. 
	If $M$ admits an orientation, then it is \emph{orientable}.
	If $M$ does not admit an orientation, then it is \emph{non-orientable}.	
\end{defn}

Orientations of manifolds let us decide whether a map is orientation preserving or reversing.

\begin{defn}
	Let $\phi:X\to M$ be a smooth map between oriented $n$--manifolds.
	If the differential $df_p:T_p X\to T_{\phi(p)} M$ is orientation preserving (resp. orientation reversing) as a map between vector spaces for every $p$ in $X$, then $\phi$ is \emph{orientation preserving} (resp. \emph{orientation reversing}) as a map.
\end{defn}