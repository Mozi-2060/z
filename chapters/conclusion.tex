We now have an algorithmic method to obtain a 4--manifold with a given boundary 3--manifold, provided that 3--manifold is orientable and given as a triangulation.
This work fits into some active areas of low dimensional topology, and there are some clear directions that we could take in the future.

The first consideration is an actual software implementation of the algorithm into a framework such as Regina: software for low dimensional topology \cite{regina}.
This is especially useful if we want to get our hands on explicit triangulations of 4--manifolds for the purposes of building a census or because we would like to compute cobordism invariants such as the Rohklin invariant.

In that same vein, the work can be refined somewhat.
For example, the handlebody built in Section \ref{sub:handlebody} is the gluing of several copies of the 3--disc, all of which have undergone two barycentric subdivisions and are thus triangulated by $21\cdot 24^2$ tetrahedra.
We then 4--thicken each tetrahedron of that handlebody with a 4--prism.
The number $P_n$ of $n$--simplices used to triangulated the $n$--prism designed in Section \ref{sub:nprism} is  $nP_{n-1} +2$, with the first few numbers being $P_1=1$, $P_2=4$, $P_3=14$, and $P_4=58$.
We quickly approach numbers of pentachora in the thousands, making some computations unfeasible without suitable simplification of triangulation.

It is also worth making modifications to the algorithm to ensure that the 4--manifold produced is simply connected, as in Theorem 5.4 of \cite{CostThur08}, or to acquire a spin structure for the 4--manifold as in Theorem 6.1 of \cite{CostThur08}.
We said before that the Stein complex was `almost' a shadow, and one stop on the route to each of these would be to explicitly turn the Stein complex acquired in Section \ref{sec:stein} into a gleamed shadow.

The final and most general avenue is the conversion of theorem proofs in smooth low--dimensional topology to algorithms on triangulations.
Much of the groundwork that went into triangulating the smooth case in this document was explored in \cite{CostThur08}, but the history of the problem has yielded a large number of smooth proofs.
The first step in our algorithm was to develop a piecewise--linear analogue to the generic proper smooth map, and this is entirely because our proof of smooth 3--manifolds bounding 4--manifolds did this as well.
Investigating other proofs that smooth 3--manifolds bound 4--manifolds may allow explicit constructions to arise, but these constructions need to be directly obtainable from a triangulation.

There are standard arguments in the literature that any closed, orientable 3--manifold has a Heegaard decomposition, and this is shown by explicitly constructing the decomposition from a triangulation of the given 3--manifold.
Combining the construction of Heegaard diagrams with proof in the smooth case of 3--manifolds bounding 4--manifolds is a promising direction for future work.
